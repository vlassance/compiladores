% !TEX encoding = UTF-8 Unicode

O programa é composto por quatro partes, explicadas abaixo de forma simplificada, pois a linguagem será definida de forma completa nos capítulos \ref{chap:bnf} e \ref{chap:wirth} nas notações BNF e Wirth, respectivamente:

\begin{itemize}
	\item Definição do programa:
		\begin{itemize}
			\item \verb$PROGRAM = IMPORTS DECLS_GLOBAIS DEF_PROCS_FUNCS DEF_MAIN.$
		\end{itemize}
	\item Inclusão de bibliotecas:
		\begin{itemize}
			\item \verb$IMPORTS = { IMPORT }.$
			\item \verb$IMPORT = `import' `<' IDENT `>' { `,' `<' IDENT `>' } `;'.$
		\end{itemize}
	\item Declaração de tipos, variáveis e constantes de escopo global:
		\begin{itemize}
			\item \verb$DECLS_GLOBAIS = { DEF_TIPO | DECL }.$
			\item \verb$DEF_TIPO = `struct' IDENT `{' { DECL } `}' `;'.$
			\item \verb$DECL = [ `const' ] TIPO IDENT [ `=' EXPR ]$\\
				\verb${ `,' IDENT [ `=' EXPR ] } `;'.$
		\end{itemize}
	\item Definição dos procedimentos e funções do programa, que não devem incluir o procedimento principal (chamado main):
		\begin{itemize}
			\item \verb$DEF_PROCS_FUNCS = { PROC | FUNC }.$
			\item \verb$FUNC = TIPO IDENT LIST_PARAMS$\\
				\verb$`{' { INSTR_SEM_RET | ( "return" EXPR ";" ) } `}'.$
			\item \verb$PROC = `void' IDENT LIST_PARAMS `{' { INSTR_SEM_RET } `}'.$
			\item \verb$LIST_PARAMS = `(' [ [ `ref' ] TIPO IDENT ]$\\
				\verb${ `,' [ `ref' ] TIPO IDENT } `)'.$
		\end{itemize}
	\item Definição do procedimento principal (chamado main) - para fins de simplificação, a comunicação entre o programa e o ambiente externo deve ser feito através de arquivos, pois não haverá passagem de parâmetros para a função main:
		\begin{itemize}
			\item \verb$DEF_MAIN = `main' `(' `)' `{' [ BLOCO ] `}'.$
		\end{itemize}
\end{itemize}
