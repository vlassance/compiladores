% !TEX encoding = UTF-8 Unicode

O programa é composto por quatro partes, explicadas abaixo de forma 
simplificada, pois a linguagem será definida de forma completa nos 
capítulos \ref{chap:bnf} e \ref{chap:wirth} nas notações \verb!BNF! e
\verb!Wirth!, respectivamente:

\begin{itemize}
    \item Definição do programa:
            
            Um programa em \verb!czar! possui em ordem obrigatória, a
            importação de bibliotecas, declaração de variáveis globais,
            definição de funções e procedimentos. O programa deve terminar
            obrigatoriamente pela declaração da função principal \verb!main!.

        \begin{itemize}
            \item \verb$PROGRAM = IMPORTS DECLS_GLOBAIS DEF_PROCS_FUNCS DEF_MAIN.$
        \end{itemize}
    \item Inclusão de bibliotecas:
        \begin{itemize}
            \item \verb$IMPORTS = { `<' IDENT `>' }.$
        \end{itemize}
    \item Declaração de tipos, variáveis e constantes de escopo global:
        \begin{itemize}
            \item \verb$DECLS_GLOBAIS = { DEF_TIPO | DECL }.$
            \item \verb$DEF_TIPO = `struct' IDENT `{' { DECL } `}'.$
            \item \verb$DECL = [ `const' ] TIPO IDENT [ `=' EXPR ]$\\
                    \verb${ `,' IDENT [ `=' EXPR ] } `;'.$
        \end{itemize}
    \item Definição dos procedimentos e funções do programa, 
        As funções não devem incluir o procedimento principal (chamado
        \verb!main!). Estas também possuem retorno final único e obrigatório. 
        \begin{itemize}
            \item \verb$DEF_PROCS_FUNCS = { PROC | FUNC }.$
            \item \verb$FUNC = TIPO IDENT LIST_PARAMS$\\
                    \verb$`{' { INSTR_SEM_RET } "return" EXPR [ ";" ] `}'.$
            \item \verb$PROC = `void' IDENT LIST_PARAMS `{' { INSTR_SEM_RET } `}'.$
            \item \verb$LIST_PARAMS = `(' [ [ `ref' ] TIPO IDENT ]$\\
                    \verb${ `,' [ `ref' ] TIPO IDENT } `)'.$
        \end{itemize}
    \item Definição do procedimento principal (chamado \verb!main!):
            
            Não existe
            passagem explícita de parâmetros para a função \verb!main!. Sendo
            que a passagem de valores para a mesma deve ocorrer por meio de
            arquivos ou pela utilização de uma função incluida por alguma
            biblioteca \emph{built-in} a ser feita. Permitindo o acesso em 
            todas as partes do código. 
        \begin{itemize}
            \item \verb$DEF_MAIN = `main' `(' `)' `{' [ BLOCO ] `}'.$
        \end{itemize}
\end{itemize}
