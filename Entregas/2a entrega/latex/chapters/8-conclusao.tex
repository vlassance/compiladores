% !TEX encoding = UTF-8 Unicode

O projeto do compilador é um projeto muito interessante, porém complexo. Desta forma, a divisão em etapas bem estruturadas permite o aprendizado e teste de cada uma das etapas. Em um primeiro momento, o foco foi no analisador léxico, o que permitiu realizar o \emph{parse} do código e transformá-lo em tokens. Para a realização do analisador, tentamos pensar em permitir o processamento das principais classes de tokens, com o intuito de entender o funcionamento de um compilador de forma prática e didática.

Já na segunda etapa, começamos definindo a linguagem de forma mais livre e geral, partindo para a criação de exemplos de códigos escritos na nossa linguagem com todos os conceitos que deveriam ser implementados. A partir da definição informal e dos exemplos de código, criamos a definição formal na notação BNF, Wirth e com Diagramas de Sintaxe, além de atualizar a lista de palavras-chave. Essa etapa nos fez refletir sobre diversos detalhes de implementação que teremos que definir para o projeto, sendo, portanto, uma etapa crucial no desenvolvimento de um compilador.

Para as próximas etapas, espera-se continuar a atualizar o código e as definições descritas nesse documento quando for necessário, visando agregar os ensinamentos das próximas aulas.
