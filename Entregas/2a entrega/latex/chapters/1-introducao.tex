% !TEX encoding = UTF-8 Unicode

Este projeto tem como objetivo a construção de um compilador de um só passo, dirigido por sintaxe, com analisador e reconhecedor sintático baseado em autômato de pilha estruturado.

Em um primeiro momento, foi definida uma linguagem de programação e identificados os tipos de átomos. Para cada átomo foi escrito uma gramática linear representativa da sua lei de formação e um reconhecedor para o átomo. Desse modo, as gramáticas assim escritas foram unidas e convertidas em um autômato finito, o qual foi transformado em um transdutor e implementado como sub-rotina, dando origem ao analisador léxico propriamente dito. Também foi criada uma função principal para chamar o analisador léxico e possibilitar o seu teste.

Nesta etapa, a sintaxe da linguagem, denonimada por nós de CZAR, foi definida formalmente a partir de uma definição informal e de exemplos de programas que criamos, misturando palavras-chave e conceitos de diferentes linguagens de programação. As três principais definições foram escritas na notação BNF\footnote{Ver http://en.wikipedia.org/wiki/Backus\_Naur\_Form}, Wirth\footnote{Ver http://en.wikipedia.org/wiki/Wirth\_syntax\_notation} e com diagramas de sintaxe.

Como material de consulta, além de sites sobre o assunto, como por exemplo um que permite verificar a definição em Wirth e criar os diagramas de sintaxe\footnote{Site: http://karmin.ch/ebnf/index}, foi utilizado o livro indicado pelo professor no começo das aulas \cite{intro-compiladores}, para pesquisa de conceitos e possíveis implementações.

O documento apresenta a seguir as respostas às questões propostas para a segunda etapa, assim como as considerações finais.
