% verso e anverso:
% \documentclass[12pt,openright,twoside,a4paper,english]{abntex2}
% apenas verso:	
\documentclass[12pt,oneside,a4paper,english]{abntex2} 

\usepackage[alf]{abntex2cite}	% Citações padrão ABNT
\usepackage{listings}
\usepackage{float}
\usepackage{cmap}				% Mapear caracteres especiais no PDF
\usepackage{lmodern}			% Usa a fonte Latin Modern			
\usepackage[T1]{fontenc}		% Selecao de codigos de fonte.
\usepackage[utf8]{inputenc}		% Codificacao do documento (conversão automática dos acentos)
\usepackage{lastpage}			% Usado pela Ficha catalográfica
\usepackage{indentfirst}		% Indenta o primeiro parágrafo de cada seção.
\usepackage{color}				% Controle das cores
\usepackage{graphicx}			% Inclusão de gráficos
\usepackage{pdfpages}
\usepackage{tikz}
\usetikzlibrary{automata,positioning}

\definecolor{blue}{RGB}{41,5,195} % alterando o aspecto da cor azul

\makeatletter
\hypersetup{
    %pagebackref=true,
    pdftitle={\@title}, 
    pdfauthor={\@author},
    pdfsubject={\@title},
    pdfcreator={\imprimirpreambulo},
    pdfkeywords={Linguagens}{Compiladores}{Definição formal da Sintaxe}, 
    colorlinks=true,       		% false: boxed links; true: colored links
    linkcolor=blue,          	% color of internal links
    citecolor=blue,        		% color of links to bibliography
    filecolor=magenta,      		% color of file links
    urlcolor=blue,
    bookmarksdepth=4
}
\makeatother

\autor{Gustavo P. Gouveia (6482819), Victor Lassance (6431325)}
\title{Relatório de Compiladores\\Segunda Etapa\\Definição formal da sintaxe da linguagem de programação \underline{CZAR}}
\orientador[Professor:]{Ricardo Luis de Azevedo da Rocha}
\preambulo{Texto apresentado à Escola Politécnica da Universidade de São Paulo como requisito para a aprovação na disciplina Linguagens e Compiladores no quinto módulo acadêmico do curso de graduação em Engenharia de Computação, junto ao Departamento de Engenharia de Computação e Sistemas Digitais (PCS).}
\instituicao{%
	Universidade de São Paulo
	\par
	Escola Politécnica
	\par
	Engenharia de Computação - Curso Cooperativo}
\local{São Paulo}
\data{2013}
\tipotrabalho{PCS2056 - Linguagens e Compiladores}

\setlength{\parindent}{1.3cm} % O tamanho do parágrafo
\setlength{\parskip}{0.2cm}  % Controle do espaçamento entre um parágrafo e outro

\makeindex

\begin{document}

\frenchspacing % Retira espaço extra obsoleto entre as frases.

\imprimirfolhaderosto

\clearpage
\begin{resumo}
	% !TEX encoding = UTF-8 Unicode
Este trabalho descreve a concepção e o desenvolvimento de um compilador utilizando a linguagem C. O escopo do compilador se limita a casos mais simples, porém simbólicos, e que servem ao aprendizado do processo de criação e teste de um compilador completo. A estrutura da linguagem escolhida para ser implementada se assemelha a própria estrutura do C, por facilidade de compreensão.

\vspace{\onelineskip}
    
\noindent
\textbf{Palavras-chaves}: Linguagens, Compiladores, Analisador Léxico.

\end{resumo}

\tableofcontents

\textual

\chapter{Introdução}
\label{chap:introducao}
	% !TEX encoding = UTF-8 Unicode

Este projeto tem como objetivo a construção de um compilador de um só passo, dirigido por sintaxe, com analisador e reconhecedor sintático baseado em autômato de pilha estruturado.

Em um primeiro momento, foi definida uma linguagem de programação e identificados os tipos de átomos. Para cada átomo foi escrito uma gramática linear representativa da sua lei de formação e um reconhecedor para o átomo. Desse modo, as gramáticas assim escritas foram unidas e convertidas em um autômato finito, o qual foi transformado em um transdutor e implementado como sub-rotina, dando origem ao analisador léxico propriamente dito. Também foi criada uma função principal para chamar o analisador léxico e possibilitar o seu teste.

Durante a segunda etapa, a sintaxe da linguagem, denonimada por nós de CZAR, foi definida formalmente a partir de uma definição informal e de exemplos de programas que criamos, misturando palavras-chave e conceitos de diferentes linguagens de programação. As três principais definições foram escritas na notação BNF\footnote{Ver http://en.wikipedia.org/wiki/Backus\_Naur\_Form}, Wirth\footnote{Ver http://en.wikipedia.org/wiki/Wirth\_syntax\_notation} e com diagramas de sintaxe.

Na terceira etapa, implementamos o módulo referente à parte sintática para a nossa linguagem. O analisador sintático construído obtém uma cadeia de \emph{tokens} proveniente do analisador léxico, e verifica se a mesma pode ser gerada pela gramática da linguagem e, com isso, constrói a árvore sintática \cite{alfred1986compilers}.

Para a quarta entrega, focamos no ambiente de execução. O compilador por nós criado tem como linguagem de saída um programa que é executado na máquina virtual conhecida como Máquina de von Neumann (MVN).

Já durante as duas últimas entregas, complementamos a especificação do código gerado pelo compilador e das rotinas do ambiente de execução da nossa linguagem de alto nível, a CZAR. Além disso, buscamos integrar as rotinas semânticas no reconhecedor sintático de forma a permitir a geração de código e finalizar o compilador.

Como material de consulta, além de sites sobre o assunto e das aulas ministradas, foi utilizado o livro indicado pelo professor no começo das aulas \cite{intro-compiladores}, para pesquisa de conceitos e possíveis implementações.

O documento apresenta a seguir o processo completo de desenvolvimento de um compilador, desde a definição formal da linguagem, passando pelo analisador léxico, reconhecedor sintático, pela definição do ambiente de execução e das rotinas semânticas, terminando com um exemplo de programa traduzido.


\chapter{Descrição Informal da Linguagem}
\label{chap:descricao-informal}
	% !TEX encoding = UTF-8 Unicode

\textbf{TODO}


\chapter{Exemplo de Programas na Linguagem}
\label{chap:exemplo-programa}
	% !TEX encoding = UTF-8 Unicode

\textbf{TODO}


\chapter{Descrição da Linguagem em BNF}
\label{chap:bnf}
	% !TEX encoding = UTF-8 Unicode

\lstinputlisting[frame=single,numbers=left,breaklines=true]{files/BNF.txt}


\chapter{Descrição da Linguagem em Wirth}
\label{chap:wirth}
	% !TEX encoding = UTF-8 Unicode

\lstinputlisting[frame=single,numbers=left,breaklines=true]{files/WIRTH.txt}


\chapter{Diagrama de Sintaxe da Linguagem}
\label{chap:diagrama}
	% !TEX encoding = UTF-8 Unicode

\textbf{TODO}


\chapter{Conjunto das Palavras Reservadas}
\label{chap:palavras-reservadas}
	% !TEX encoding = UTF-8 Unicode

\textbf{TODO}


\chapter{Considerações Finais}
\label{chap:conclusao}
	% !TEX encoding = UTF-8 Unicode

O projeto do compilador é um projeto muito interessante, porém complexo. Desta forma, a divisão em etapas bem estruturadas permite o aprendizado e teste de cada uma das etapas. Em um primeiro momento, o foco foi no analisador léxico, o que permitiu realizar o \emph{parse} do código e transformá-lo em tokens. Para a realização do analisador, tentamos pensar em permitir o processamento das principais classes de tokens, com o intuito de entender o funcionamento de um compilador de forma prática e didática.

Já na segunda etapa, \textbf{TODO:Victor}

Para as próximas etapas, espera-se atualizar o analisador léxico quando for necessário, visando agregar os ensinamentos das próximas aulas.


\bibliography{bibliografia}

\end{document}
