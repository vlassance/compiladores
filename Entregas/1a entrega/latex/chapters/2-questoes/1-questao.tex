% !TEX encoding = UTF-8 Unicode

\textbf{Quais são as funções do analisador léxico nos compiladores e interpretadores?}

O  analisador léxico atua como uma interface entre o reconhecedor sintático, que forma, normalmente, o núcleo do compilador, e o texto de entrada, convertendo a sequência de caracteres de que este se constitui em uma sequência de átomos.

Para a consecução de seus objetivos, o analisador léxico executa usualmente uma série de funções, todas de grande importância como infraestrutura para a operação das partes do compilador mais ligadas à tradução propriamente dita do texto-fonte. As principais funções são listadas abaixo:

\begin{itemize}

	\item Extração e Classificação de Átomos;
	\begin{itemize}
		\item Principal funcionalidade do analisador;
		\item As classes de átomos mais usuais: identificadores, palavras reservadas, números inteiros sem sinal, números reais, strings, sinais de pontuação e de operação, caracteres especiais, símbolos compostos de dois ou mais caracteres especiais e comentários.
	\end{itemize}
	
	\item Eliminação de Delimitadores e Comentários;
	
	\item Conversão numérica;
	\begin{itemize}
		\item Conversão numérica de notações diversas em uma forma interna de representação para manipulação de pelos demais módulos do compilador.
	\end{itemize}
	
	\item Tratamento de Identificadores;
	\begin{itemize}
		\item Tratamento com auxílio de uma tabela de símbolos.
	\end{itemize}
	
	\item Identificação de Palavras Reservadas;
	\begin{itemize}
		\item Verificar se cada identificador reconhecido pertence a um conjunto de identificadores especiais.
	\end{itemize}
	
	\item Recuperação de Erros;
	
	\item Listagens;
	\begin{itemize}
		\item Geração de listagens do texto-fonte.
	\end{itemize}
	
	\item Geração de Tabelas de Referências Cruzadas;
	\begin{itemize}
		\item Geração de listagem indicativa dos símbolos encontrados, com menção à localização de todas as suas ocorrências no texto do programa-fonte.
	\end{itemize}
	
	\item Definição e Expansão de Macros;
	\begin{itemize}
		\item Pode ser realizado em um pré-processamento ou no analisador léxico. No caso do analisador, deve-se haver uma comunicação entres os analisadores léxico e sintático.
	\end{itemize}
	
	\item Interação com o sistema de arquivos;
	
	\item Compilação Condicional;
	
	\item Controles de Listagens.
	\begin{itemize}
		\item São os comandos que permitem ao programador que ligue e desligue opções de listagem, de coleta de símbolos em tabelas de referência cruzadas, de geração, e impressão de tais tabelas, de impressão de tabelas de símbolos do programa compilador, de tabulação e formatação das saídas impressas do programa-fonte.
	\end{itemize}

\end{itemize}