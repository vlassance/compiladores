% !TEX encoding = UTF-8 Unicode

Para realizar o teste do compilador e verificar sua execução, foi criado um arquivo README, adicionado ao código, que explica como testar o léxico somente, o compilador completo e o compilador com a execução no interpretador \emph{RNA}. Para cada um deles, basta rodar \textbf{make clean} e \textbf{make <comando>}, <comando> = lextest, compilertest e runrna. Todos executarão tendo como base o arquivo examples/simprolog.pro.

Também cabe relembrar que deve ser utilizada a versão do interpretador \emph{RNA} que está anexada ao código, com as mudanças propostas no Capítulo~\ref{chap:linguagem-rna}.

Segue abaixo dois exemplos de programas executados e seus respectivos resultados.

\section{Programa 1}

Abaixo, segue um exemplo de programa rodado:

\lstinputlisting[frame=single,breaklines=true,morekeywords={not,or,eps},basicstyle=\tiny,numbers=left]{files/simprolog.pro}

O resultado do código executado pelo interpretador \emph{RNA} segue abaixo:

\lstinputlisting[frame=single,breaklines=true]{files/res_simprolog.txt}

\section{Programa 2}

Abaixo, segue um exemplo de programa rodado:

\lstinputlisting[frame=single,breaklines=true,morekeywords={not,or,eps},basicstyle=\tiny,numbers=left]{files/simprolog_2.pro}

O resultado do código executado pelo interpretador \emph{RNA} segue abaixo:

\lstinputlisting[frame=single,breaklines=true]{files/res_simprolog_2.txt}