% verso e anverso:
% \documentclass[12pt,openright,twoside,a4paper,english]{abntex2}
% apenas verso:	
\documentclass[12pt,oneside,a4paper,english]{abntex2} 

\usepackage[alf]{abntex2cite}	% Citações padrão ABNT
\usepackage{listings}
\usepackage{float}
\usepackage{cmap}				% Mapear caracteres especiais no PDF
\usepackage{lmodern}			% Usa a fonte Latin Modern			
\usepackage[T1]{fontenc}		% Selecao de codigos de fonte.
\usepackage[utf8]{inputenc}		% Codificacao do documento (conversão automática dos acentos)
\usepackage{lastpage}			% Usado pela Ficha catalográfica
\usepackage{indentfirst}		% Indenta o primeiro parágrafo de cada seção.
\usepackage{color}				% Controle das cores
\usepackage{graphicx}			% Inclusão de gráficos
\usepackage{pdfpages}
\usepackage{tikz}
\usetikzlibrary{automata,positioning}
\usepackage{mathtools}

\definecolor{blue}{RGB}{41,5,195} % alterando o aspecto da cor azul

\makeatletter
\hypersetup{
    %pagebackref=true,
    pdftitle={\@title}, 
    pdfauthor={\@author},
    pdfsubject={\@title},
    pdfcreator={\imprimirpreambulo},
    pdfkeywords={Linguagens}{Compiladores}{Tradução dos Comandos}, 
    colorlinks=true,       		% false: boxed links; true: colored links
    linkcolor=blue,          	% color of internal links
    citecolor=blue,        		% color of links to bibliography
    filecolor=magenta,      		% color of file links
    urlcolor=blue,
    bookmarksdepth=4
}
\makeatother

\autor{Gustavo P. Gouveia (6482819), Victor Lassance (6431325)}
\title{Relatório de Compiladores\\Quinta Etapa\\Tradução dos Comandos\\Linguagem de programação \underline{CZAR}}
\orientador[Professor:]{Ricardo Luis de Azevedo da Rocha}
\preambulo{Texto apresentado à Escola Politécnica da Universidade de São Paulo como requisito para a aprovação na disciplina Linguagens e Compiladores no quinto módulo acadêmico do curso de graduação em Engenharia de Computação, junto ao Departamento de Engenharia de Computação e Sistemas Digitais (PCS).}
\instituicao{%
	Universidade de São Paulo
	\par
	Escola Politécnica
	\par
	Engenharia de Computação - Curso Cooperativo}
\local{São Paulo}
\data{2013}
\tipotrabalho{PCS2056 - Linguagens e Compiladores}

\setlength{\parindent}{1.3cm} % O tamanho do parágrafo
\setlength{\parskip}{0.2cm}  % Controle do espaçamento entre um parágrafo e outro
\lstset{basicstyle=\footnotesize\ttfamily}
\makeindex

\begin{document}

\frenchspacing % Retira espaço extra obsoleto entre as frases.

\imprimirfolhaderosto

\tableofcontents

\textual

\chapter{Introdução}
\label{chap:introducao}
	% !TEX encoding = UTF-8 Unicode

Este projeto tem como objetivo a construção de um compilador de um só passo, dirigido por sintaxe, com analisador e reconhecedor sintático baseado em autômato de pilha estruturado.

Em um primeiro momento, foi definida uma linguagem de programação e identificados os tipos de átomos. Para cada átomo foi escrito uma gramática linear representativa da sua lei de formação e um reconhecedor para o átomo. Desse modo, as gramáticas assim escritas foram unidas e convertidas em um autômato finito, o qual foi transformado em um transdutor e implementado como sub-rotina, dando origem ao analisador léxico propriamente dito. Também foi criada uma função principal para chamar o analisador léxico e possibilitar o seu teste.

Durante a segunda etapa, a sintaxe da linguagem, denonimada por nós de CZAR, foi definida formalmente a partir de uma definição informal e de exemplos de programas que criamos, misturando palavras-chave e conceitos de diferentes linguagens de programação. As três principais definições foram escritas na notação BNF\footnote{Ver http://en.wikipedia.org/wiki/Backus\_Naur\_Form}, Wirth\footnote{Ver http://en.wikipedia.org/wiki/Wirth\_syntax\_notation} e com diagramas de sintaxe.

Na terceira etapa, implementamos o módulo referente à parte sintática para a nossa linguagem. O analisador sintático construído obtém uma cadeia de \emph{tokens} proveniente do analisador léxico, e verifica se a mesma pode ser gerada pela gramática da linguagem e, com isso, constrói a árvore sintática \cite{alfred1986compilers}.

Para a quarta entrega, focamos no ambiente de execução. O compilador por nós criado terá como linguagem de saída um programa que será executado na máquina virtual conhecida como Máquina de von Neumann (MVN).

Para a entrega atual, buscamos completar a especificação do código gerado pelo compilador e das rotinas do ambiente de execução da nossa linguagem de alto nível, a CZAR.

Como material de consulta, além de sites sobre o assunto e das aulas ministradas, foi utilizado o livro indicado pelo professor no começo das aulas \cite{intro-compiladores}, para pesquisa de conceitos e possíveis implementações.

O documento apresenta a seguir o que foi solicitado na quinta etapa.


\chapter{Tradução de estruturas de controle de fluxo}
\label{chap:traducao-estruturas-controle}
	% !TEX encoding = UTF-8 Unicode

Será apresentado nas próximas seções, as traduções das estruturas de controle de fluxo que constam na nossa linguagem e foram solicitadas para essa entrega, entre elas as estruturas if, if-else e while.

Cabe ressaltar que foram utilizadas simbologias nas traduções que serão substituídas pelo compilador no momento da geração de código. Uma dessas marcações é os dois pontos no começo de uma linha que significa que os comandos devem ser colocados no início do código gerado. Outra simbologia criada é da forma {XN}, onde X representa uma letra maiúscula qualquer e N é o índice da instância dentro do tipo de marcação X. As opções para X são as seguintes:

\begin{itemize}
	\item \verb={C0}, {C1}, ...=: Conjunto de comandos
	\item \verb={R0}, {R1}, ...=: Referência
	\item \verb={L0}, {L1}, ...=: Label ou rótulo de uma instrução criados
            e exportados pelo código 
\end{itemize}

Há também a marcação \verb={N}=, utilizada para denotar que a primeira instrução do código subsequente ao comando atual deve ser adicionada no lugar da marcação. Estamos considerando substituir sempre a marcação \verb={N}= por uma instrução simples que só sirva para simplificar, como por exemplo somar zero ao acumulador.

Conceitos da pilha aritmética são utilizados para o cálculo de expressões
booleanas, no \autoref{sec:expre} explicações mais detalhadas são apresentadas. 

\section{Estrutura de controle de fluxo: IF}
\label{sec:if}

\lstinputlisting[frame=single,numbers=left,breaklines=true,morekeywords={JP,JZ,JN,LV,LD,MM,SC,RS,HM,GD,PD,OS}]{precompiled/if.pre}

\section{Estrutura de controle de fluxo: IF-ELSE}
\label{sec:if-else}

\lstinputlisting[frame=single,numbers=left,breaklines=true,morekeywords={JP,JZ,JN,LV,LD,MM,SC,RS,HM,GD,PD,OS}]{precompiled/ifelse.pre}

\section{Estrutura de controle de fluxo: WHILE}
\label{sec:while}

\lstinputlisting[frame=single,numbers=left,breaklines=true,morekeywords={JP,JZ,JN,LV,LD,MM,SC,RS,HM,GD,PD,OS}]{precompiled/while.pre}


\chapter{Tradução de comandos imperativos}
\label{chap:traducao-comandos-imperativos}
	% !TEX encoding = UTF-8 Unicode

Esse capítulo explica as traduções dos comandos imperativos que constam na nossa linguagem e foram solicitadas para essa entrega, entre os quais os comandos de atribuição de valor, leitura da entrada padrão, impressão na saída padrão e chamada de subrotinas, associado à definicão de novas subrotinas. As mesmas definições das marcações explicadas no Capítulo~\ref{chap:traducao-estruturas-controle} são válidas para as traduções a seguir.

\section{Atribuição de valor}
\label{sec:atribuicao-valor}

\lstinputlisting[frame=single,numbers=left,breaklines=true,morekeywords={JP,JZ,JN,LV,LD,MM,SC,RS,HM,GD,PD,OS}]{precompiled/atribV.pre}

\section{Comando de leitura}
\label{sec:leitura}

\lstinputlisting[frame=single,numbers=left,breaklines=true,morekeywords={JP,JZ,JN,LV,LD,MM,SC,RS,HM,GD,PD,OS}]{precompiled/read.pre}

\section{Comando de impressão}
\label{sec:impressao}

\lstinputlisting[frame=single,numbers=left,breaklines=true,morekeywords={JP,JZ,JN,LV,LD,MM,SC,RS,HM,GD,PD,OS}]{precompiled/write.pre}

\section{Definição e chamada de subrotinas}
\label{sec:subrotinas}

No caso da definição de subrotinas, a tradução fica a seguinte: 

\lstinputlisting[frame=single,numbers=left,breaklines=true,morekeywords={JP,JZ,JN,LV,LD,MM,SC,RS,HM,GD,PD,OS}]{precompiled/function.pre}

Vale salientar que as funcoes que tratam a pilha de registro de ativação foram
modificadas completamente para integração mais transparente na implementacao da
função. 

Já quando é identificada a chamada de uma subrotina já declarada, a seguinte tradução é utilizada:

\lstinputlisting[frame=single,numbers=left,breaklines=true,morekeywords={JP,JZ,JN,LV,LD,MM,SC,RS,HM,GD,PD,OS}]{precompiled/call.pre}

	
\chapter{Cálculo de expressões aritméticas e booleanas}
\label{chap:calculo-expressoes}
	% !TEX encoding = UTF-8 Unicode

Cálculo de expressões - código alinhavado.


\chapter{Arrays e Structs}
\label{chap:arrays-structs}
	% !TEX encoding = UTF-8 Unicode
Em \emph{CZAR} \textbf{não} existem \emph{Arrays} de tamanho dinâmico e sua criação está
limitada à declaração. Sendo assim, suas dimensões internas são conhecidas pelo
compilador a todo momento e seu cálculo de posição é facilitado e feito em
tempo de compilação. 

\begin{lstlisting}[frame=single,numbers=left,breaklines=true, language=C]
int i;

/* 
* Array int[4][3][2]: 
*
* [
*    [[0, 0], [0, 0], [0, 0]], 
*    [[0, 0], [0, 0], [0, 0]], 
*    [[0, 0], [0, 0], [0, 0]], 
*    [[0, 0], [0, 0], [0, 0]]
* ]
*
* Preenchendo com:
* ----------------
* decl int i;
* decl int j;
* decl int k;
* decl int l; 
* set i = 0;
* set j = 0;
* while (j < 4) {
*   set k = 0;
*   while (k < 3) {
*     set l = 0;
*     while (l < 2) {
*       set array_ex[j][k][l] = i;
*       set i = i + 1;
*       set l = l + 1;
*     }
*     set k = k + 1;
*   }
*   set j = j + 1;
* }
*
* Temos:
* ------
* [
*    [[0, 1], [2, 3], [4, 5]], 
*    [[6, 7], [8, 9], [10, 11]], 
*    [[12, 13], [14, 15], [16, 17]], 
*    [[18, 19], [20, 21], [22, 23]]
* ]
* ou:
* ---
* [
*   0, 1, 2, 3, 4, 5, 6, 7, 8, 9, 10,
*   10, 11, 12, 13, 14, 15, 16, 17, 18, 
*   19, 20, 21, 22, 23
* ]
*
*
*
*
*/
acc = acumulado[n_dimensoes - 1] = 1; // maybe long 1L 
for (i = n_dimensoes-2; i >= 0; i--) {
   acc = acumulado[i+1] = dimensoes[i+1] * acc;
}
/* 
   acumulado = [6, 2, 1];
*/
for (i = 0; i < n_dimensoes; i++) {
    acumulado[i] = acumulado[i] * size_cell; // celulas podem ter 
                                             // tamanho variavel
}
for (i = 0; i < n_dimensoes; i++) {
    fprints(str, " LV =%d ", acumulado[i]);
    cpy_to_lines_of_code(str);
    fprints(str, " * ARR_DIM_%d", acumulado[i]);
    cpy_to_lines_of_code(str);
    fprints(str, " +  ADDRS_ACCUMULATOR");
    cpy_to_lines_of_code(str);
    fprints(str, " MM ADDRS_ACCUMULATOR");
    cpy_to_lines_of_code(str);
}
\end{lstlisting}

O cálculo de \emph{structs} é resolvido em tempo de compilação. Uma vez que o
tamanho de cada parte da estrutura é conhecida em tempo de compilação, é
possível se fazer toda a aritmética de acesso via programação em \emph{C}. 

\begin{lstlisting}[frame=single,numbers=left,breaklines=true, language=C]
    int deslocamento_para_celula(struct_struct* vi_struct, int cell_to_access) {
        int sum_up_to_ptr = 0;
        for (i = 0; i < cell_to_access; i++) {
            sum_up_to_ptr += vi_struct->sizes[i];
        }
        return sum_up_to_ptr;
    }
\end{lstlisting}


\chapter{Exemplo de programa traduzido}
\label{chap:caracteristicas-gerais}
	\input{chapters/6-exemplo-programa}

\postextual

\bibliography{bibliografia}

\begin{apendicesenv} % Inicia os apêndices

	% Imprime uma página indicando o início dos apêndices
	\partapendices
	
	\chapter{Biblioteca auxiliar \emph{std}}
	\label{app:std}
	
	\lstinputlisting[frame=single,numbers=left,breaklines=true,morekeywords={JP,JZ,JN,LV,LD,MM,SC,RS,HM,GD,PD,OS}]{example_compiled/std.asm}
	
	\chapter{Biblioteca auxiliar \emph{stdio}}
	\label{app:stdio}
	
	\lstinputlisting[frame=single,numbers=left,breaklines=true,morekeywords={JP,JZ,JN,LV,LD,MM,SC,RS,HM,GD,PD,OS}]{example_compiled/stdio.asm}

\end{apendicesenv}

\end{document}
