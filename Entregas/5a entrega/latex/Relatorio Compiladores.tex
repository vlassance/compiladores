% verso e anverso:
% \documentclass[12pt,openright,twoside,a4paper,english]{abntex2}
% apenas verso:	
\documentclass[12pt,oneside,a4paper,english]{abntex2} 

\usepackage[alf]{abntex2cite}	% Citações padrão ABNT
\usepackage{listings}
\usepackage{float}
\usepackage{cmap}				% Mapear caracteres especiais no PDF
\usepackage{lmodern}			% Usa a fonte Latin Modern			
\usepackage[T1]{fontenc}		% Selecao de codigos de fonte.
\usepackage[utf8]{inputenc}		% Codificacao do documento (conversão automática dos acentos)
\usepackage{lastpage}			% Usado pela Ficha catalográfica
\usepackage{indentfirst}		% Indenta o primeiro parágrafo de cada seção.
\usepackage{color}				% Controle das cores
\usepackage{graphicx}			% Inclusão de gráficos
\usepackage{pdfpages}
\usepackage{tikz}
\usetikzlibrary{automata,positioning}
\usepackage{mathtools}

\definecolor{blue}{RGB}{41,5,195} % alterando o aspecto da cor azul

\makeatletter
\hypersetup{
    %pagebackref=true,
    pdftitle={\@title}, 
    pdfauthor={\@author},
    pdfsubject={\@title},
    pdfcreator={\imprimirpreambulo},
    pdfkeywords={Linguagens}{Compiladores}{Tradução dos Comandos}, 
    colorlinks=true,       		% false: boxed links; true: colored links
    linkcolor=blue,          	% color of internal links
    citecolor=blue,        		% color of links to bibliography
    filecolor=magenta,      		% color of file links
    urlcolor=blue,
    bookmarksdepth=4
}
\makeatother

\autor{Gustavo P. Gouveia (6482819), Victor Lassance (6431325)}
\title{Relatório de Compiladores\\Quinta Etapa\\Tradução dos Comandos\\Linguagem de programação \underline{CZAR}}
\orientador[Professor:]{Ricardo Luis de Azevedo da Rocha}
\preambulo{Texto apresentado à Escola Politécnica da Universidade de São Paulo como requisito para a aprovação na disciplina Linguagens e Compiladores no quinto módulo acadêmico do curso de graduação em Engenharia de Computação, junto ao Departamento de Engenharia de Computação e Sistemas Digitais (PCS).}
\instituicao{%
	Universidade de São Paulo
	\par
	Escola Politécnica
	\par
	Engenharia de Computação - Curso Cooperativo}
\local{São Paulo}
\data{2013}
\tipotrabalho{PCS2056 - Linguagens e Compiladores}

\setlength{\parindent}{1.3cm} % O tamanho do parágrafo
\setlength{\parskip}{0.2cm}  % Controle do espaçamento entre um parágrafo e outro

\makeindex

\begin{document}

\frenchspacing % Retira espaço extra obsoleto entre as frases.

\imprimirfolhaderosto

\tableofcontents

\textual

\chapter{Introdução}
\label{chap:introducao}
	% !TEX encoding = UTF-8 Unicode

Este projeto tem como objetivo a construção de um compilador de um só passo, dirigido por sintaxe, com analisador e reconhecedor sintático baseado em autômato de pilha estruturado.

Em um primeiro momento, foi definida uma linguagem de programação e identificados os tipos de átomos. Para cada átomo foi escrito uma gramática linear representativa da sua lei de formação e um reconhecedor para o átomo. Desse modo, as gramáticas assim escritas foram unidas e convertidas em um autômato finito, o qual foi transformado em um transdutor e implementado como sub-rotina, dando origem ao analisador léxico propriamente dito. Também foi criada uma função principal para chamar o analisador léxico e possibilitar o seu teste.

Durante a segunda etapa, a sintaxe da linguagem, denonimada por nós de CZAR, foi definida formalmente a partir de uma definição informal e de exemplos de programas que criamos, misturando palavras-chave e conceitos de diferentes linguagens de programação. As três principais definições foram escritas na notação BNF\footnote{Ver http://en.wikipedia.org/wiki/Backus\_Naur\_Form}, Wirth\footnote{Ver http://en.wikipedia.org/wiki/Wirth\_syntax\_notation} e com diagramas de sintaxe.

Na terceira etapa, implementamos o módulo referente à parte sintática para a nossa linguagem. O analisador sintático construído obtém uma cadeia de \emph{tokens} proveniente do analisador léxico, e verifica se a mesma pode ser gerada pela gramática da linguagem e, com isso, constrói a árvore sintática \cite{alfred1986compilers}.

Para a quarta entrega, focamos no ambiente de execução. O compilador por nós criado tem como linguagem de saída um programa que é executado na máquina virtual conhecida como Máquina de von Neumann (MVN).

Já durante as duas últimas entregas, complementamos a especificação do código gerado pelo compilador e das rotinas do ambiente de execução da nossa linguagem de alto nível, a CZAR. Além disso, buscamos integrar as rotinas semânticas no reconhecedor sintático de forma a permitir a geração de código e finalizar o compilador.

Como material de consulta, além de sites sobre o assunto e das aulas ministradas, foi utilizado o livro indicado pelo professor no começo das aulas \cite{intro-compiladores}, para pesquisa de conceitos e possíveis implementações.

O documento apresenta a seguir o processo completo de desenvolvimento de um compilador, desde a definição formal da linguagem, passando pelo analisador léxico, reconhecedor sintático, pela definição do ambiente de execução e das rotinas semânticas, terminando com um exemplo de programa traduzido.


\chapter{Tradução de estruturas de controle de fluxo}
\label{chap:traducao-estruturas-controle}
	% !TEX encoding = UTF-8 Unicode

Tradução de estruturas de controle de fluxo

- Desvio

- If-then (obrigatório)

- If-then-else (obrigatório)

- While (obrigatório)

- Do-until

Falar que o comando {N} pode ser substituído por +0, por exemplo, para simplificar o processo e tornar o código sequencial.


\chapter{Tradução de comandos imperativos}
\label{chap:traducao-comandos-imperativos}
	% !TEX encoding = UTF-8 Unicode

Tradução de comandos imperativos

- Atribuição de valor (obrigatório)

- Leitura (entrada) (obrigatório)

- Impressão (saída) (obrigatório)

- Chamada de subrotina (obrigatório)

	
\chapter{Cálculo de expressões aritméticas e booleanas}
\label{chap:calculo-expressoes}
	% !TEX encoding = UTF-8 Unicode

Cálculo de expressões - código alinhavado.


\chapter{Exemplo de programa traduzido}
\label{chap:caracteristicas-gerais}
	% !TEX encoding = UTF-8 Unicode

A fim de demonstrar tudo o que foi pensado como a maneira de traduzir os comandos de alto nível da nossa linguagem CZAR, nós traduzimos um programa simples de fatorial que permite visualizar e testar a nossa tradução.

Para isso, apresentamos o exemplo de programa escrito em três diferentes linguagens: (i) na nossa linguagem de alto nível CZAR; (ii) tradução para linguagem de máquina, utilizando as bibliotecas complementares \emph{std} e \emph{stdio}; (iii) tradução para linguagem de saída MVN.

Adicionamos as bibliotecas \emph{std} e \emph{stdio} como apêndices (ver Apêndice~\ref{app:std} e~\ref{app:stdio}) desse documento para consulta sobre o que já foi efetivamente desenvolvido.

\section{Exemplo de programa fatorial na linguagem de alto nível}
\label{sec:alto-nivel}

\lstinputlisting[frame=single,numbers=left,breaklines=true,morekeywords={main,const,int,char,string,void,return,for,struct,if,ref,float,else,and,or,not,true,false}]{example_compiled/exemplo_fatorial.czar}

\section{Tradução do programa fatorial para linguagem de máquina}
\label{sec:traducao-asm}

\lstinputlisting[frame=single,numbers=left,breaklines=true,morekeywords={JP,JZ,JN,LV,LD,MM,SC,RS,HM,GD,PD,OS}]{example_compiled/exemplo_fatorial_gen.asm}

\section{Tradução do programa fatorial para linguagem de saída MVN}
\label{sec:traducao-mvn}

\lstinputlisting[frame=single,numbers=left,breaklines=true]{example_compiled/exemplo_fatorial_gen.mvn}


\postextual

\bibliography{bibliografia}

\begin{apendicesenv} % Inicia os apêndices

	% Imprime uma página indicando o início dos apêndices
	\partapendices
	
	\chapter{Biblioteca auxiliar \emph{std}}
	\label{app:std}
	
	\lstinputlisting[frame=single,numbers=left,breaklines=true,morekeywords={JP,JZ,JN,LV,LD,MM,SC,RS,HM,GD,PD,OS}]{example_compiled/std.asm}
	
	\chapter{Biblioteca auxiliar \emph{stdio}}
	\label{app:stdio}
	
	\lstinputlisting[frame=single,numbers=left,breaklines=true,morekeywords={JP,JZ,JN,LV,LD,MM,SC,RS,HM,GD,PD,OS}]{example_compiled/stdio.asm}

\end{apendicesenv}

\end{document}
