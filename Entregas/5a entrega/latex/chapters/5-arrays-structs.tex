% !TEX encoding = UTF-8 Unicode
Em \emph{CZAR} \textbf{não} existem \emph{Arrays} de tamanho dinâmico e sua criação está
limitada à declaração. Sendo assim, suas dimensões internas são conhecidas pelo
compilador a todo momento e seu cálculo de posição é facilitado e feito em
tempo de compilação. 

\begin{lstlisting}[frame=single,numbers=left,breaklines=true, language=C]
int i;

/* 
* Array int[4][3][2]: 
*
* [
*    [[0, 0], [0, 0], [0, 0]], 
*    [[0, 0], [0, 0], [0, 0]], 
*    [[0, 0], [0, 0], [0, 0]], 
*    [[0, 0], [0, 0], [0, 0]]
* ]
*
* Preenchendo com:
* ----------------
* decl int i;
* decl int j;
* decl int k;
* decl int l; 
* set i = 0;
* set j = 0;
* while (j < 4) {
*   set k = 0;
*   while (k < 3) {
*     set l = 0;
*     while (l < 2) {
*       set array_ex[j][k][l] = i;
*       set i = i + 1;
*       set l = l + 1;
*     }
*     set k = k + 1;
*   }
*   set j = j + 1;
* }
*
* Temos:
* ------
* [
*    [[0, 1], [2, 3], [4, 5]], 
*    [[6, 7], [8, 9], [10, 11]], 
*    [[12, 13], [14, 15], [16, 17]], 
*    [[18, 19], [20, 21], [22, 23]]
* ]
* ou:
* ---
* [
*   0, 1, 2, 3, 4, 5, 6, 7, 8, 9, 10,
*   10, 11, 12, 13, 14, 15, 16, 17, 18, 
*   19, 20, 21, 22, 23
* ]
*
*
*
*
*/
acc = acumulado[n_dimensoes - 1] = 1; // maybe long 1L 
for (i = n_dimensoes-2; i >= 0; i--) {
   acc = acumulado[i+1] = dimensoes[i+1] * acc;
}
/* 
   acumulado = [6, 2, 1];
*/
for (i = 0; i < n_dimensoes; i++) {
    acumulado[i] = acumulado[i] * size_cell; // celulas podem ter 
                                             // tamanho variavel
}
for (i = 0; i < n_dimensoes; i++) {
    fprints(str, " LV =%d ", acumulado[i]);
    cpy_to_lines_of_code(str);
    fprints(str, " * ARR_DIM_%d", acumulado[i]);
    cpy_to_lines_of_code(str);
    fprints(str, " +  ADDRS_ACCUMULATOR");
    cpy_to_lines_of_code(str);
    fprints(str, " MM ADDRS_ACCUMULATOR");
    cpy_to_lines_of_code(str);
}
\end{lstlisting}

O cálculo de \emph{structs} é resolvido em tempo de compilação. Uma vez que o
tamanho de cada parte da estrutura é conhecida em tempo de compilação, é
possível se fazer toda a aritmética de acesso via programação em \emph{C}. 

\begin{lstlisting}[frame=single,numbers=left,breaklines=true, language=C]
    int deslocamento_para_celula(struct_struct* vi_struct, int cell_to_access) {
        int sum_up_to_ptr = 0;
        for (i = 0; i < cell_to_access; i++) {
            sum_up_to_ptr += vi_struct->sizes[i];
        }
        return sum_up_to_ptr;
    }
\end{lstlisting}
