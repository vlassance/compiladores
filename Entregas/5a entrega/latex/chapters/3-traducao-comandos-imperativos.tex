% !TEX encoding = UTF-8 Unicode

Esse capítulo explica as traduções dos comandos imperativos que constam na nossa linguagem e foram solicitadas para essa entrega, entre os quais os comandos de atribuição de valor, leitura da entrada padrão, impressão na saída padrão e chamada de subrotinas, associado à definicão de novas subrotinas. As mesmas definições das marcações explicadas no Capítulo~\ref{chap:traducao-estruturas-controle} são válidas para as traduções a seguir.

\section{Atribuição de valor}
\label{sec:atribuicao-valor}

\lstinputlisting[frame=single,numbers=left,breaklines=true,morekeywords={JP,JZ,JN,LV,LD,MM,SC,RS,HM,GD,PD,OS}]{precompiled/atribV.pre}

\section{Comando de leitura}
\label{sec:leitura}

\lstinputlisting[frame=single,numbers=left,breaklines=true,morekeywords={JP,JZ,JN,LV,LD,MM,SC,RS,HM,GD,PD,OS}]{precompiled/read.pre}

\section{Comando de impressão}
\label{sec:impressao}

\lstinputlisting[frame=single,numbers=left,breaklines=true,morekeywords={JP,JZ,JN,LV,LD,MM,SC,RS,HM,GD,PD,OS}]{precompiled/write.pre}

\section{Definição e chamada de subrotinas}
\label{sec:subrotinas}

No caso da definição de subrotinas, a tradução fica a seguinte:

\lstinputlisting[frame=single,numbers=left,breaklines=true,morekeywords={JP,JZ,JN,LV,LD,MM,SC,RS,HM,GD,PD,OS}]{precompiled/function.pre}

Já quando é identificada a chamada de uma subrotina já declarada, a seguinte tradução é utilizada:

\lstinputlisting[frame=single,numbers=left,breaklines=true,morekeywords={JP,JZ,JN,LV,LD,MM,SC,RS,HM,GD,PD,OS}]{precompiled/call.pre}
