% verso e anverso:
% \documentclass[12pt,openright,twoside,a4paper,english]{abntex2}
% apenas verso:	
\documentclass[12pt,oneside,a4paper,english]{abntex2} 

\usepackage[alf]{abntex2cite}	% Citações padrão ABNT
\usepackage{listings}
\usepackage{float}
\usepackage{cmap}				% Mapear caracteres especiais no PDF
\usepackage{lmodern}			% Usa a fonte Latin Modern			
\usepackage[T1]{fontenc}		% Selecao de codigos de fonte.
\usepackage[utf8]{inputenc}		% Codificacao do documento (conversão automática dos acentos)
\usepackage{lastpage}			% Usado pela Ficha catalográfica
\usepackage{indentfirst}		% Indenta o primeiro parágrafo de cada seção.
\usepackage{color}				% Controle das cores
\usepackage{graphicx}			% Inclusão de gráficos
\usepackage{pdfpages}
\usepackage{tikz}
\usetikzlibrary{automata,positioning}
\usepackage{mathtools}

\definecolor{blue}{RGB}{41,5,195} % alterando o aspecto da cor azul

\makeatletter
\hypersetup{
    %pagebackref=true,
    pdftitle={\@title}, 
    pdfauthor={\@author},
    pdfsubject={\@title},
    pdfcreator={\imprimirpreambulo},
    pdfkeywords={Linguagens}{Compiladores}{Especificação do Ambiente de Execução}, 
    colorlinks=true,       		% false: boxed links; true: colored links
    linkcolor=blue,          	% color of internal links
    citecolor=blue,        		% color of links to bibliography
    filecolor=magenta,      		% color of file links
    urlcolor=blue,
    bookmarksdepth=4
}
\makeatother

\autor{Gustavo P. Gouveia (6482819), Victor Lassance (6431325)}
\title{Relatório de Compiladores\\Quarta Etapa\\Especificação do Ambiente de Execução\\Linguagem de programação \underline{CZAR}}
\orientador[Professor:]{Ricardo Luis de Azevedo da Rocha}
\preambulo{Texto apresentado à Escola Politécnica da Universidade de São Paulo como requisito para a aprovação na disciplina Linguagens e Compiladores no quinto módulo acadêmico do curso de graduação em Engenharia de Computação, junto ao Departamento de Engenharia de Computação e Sistemas Digitais (PCS).}
\instituicao{%
	Universidade de São Paulo
	\par
	Escola Politécnica
	\par
	Engenharia de Computação - Curso Cooperativo}
\local{São Paulo}
\data{2013}
\tipotrabalho{PCS2056 - Linguagens e Compiladores}

\setlength{\parindent}{1.3cm} % O tamanho do parágrafo
\setlength{\parskip}{0.2cm}  % Controle do espaçamento entre um parágrafo e outro

\makeindex

\begin{document}

\frenchspacing % Retira espaço extra obsoleto entre as frases.

\imprimirfolhaderosto

\tableofcontents

\textual

\chapter{Introdução}
\label{chap:introducao}
	% !TEX encoding = UTF-8 Unicode

Este projeto tem como objetivo a construção de um compilador de um só passo, dirigido por sintaxe, com analisador e reconhecedor sintático baseado em autômato de pilha estruturado.

Em um primeiro momento, foi definida uma linguagem de programação e identificados os tipos de átomos. Para cada átomo foi escrito uma gramática linear representativa da sua lei de formação e um reconhecedor para o átomo. Desse modo, as gramáticas assim escritas foram unidas e convertidas em um autômato finito, o qual foi transformado em um transdutor e implementado como sub-rotina, dando origem ao analisador léxico propriamente dito. Também foi criada uma função principal para chamar o analisador léxico e possibilitar o seu teste.

Durante a segunda etapa, a sintaxe da linguagem, denonimada por nós de CZAR, foi definida formalmente a partir de uma definição informal e de exemplos de programas que criamos, misturando palavras-chave e conceitos de diferentes linguagens de programação. As três principais definições foram escritas na notação BNF\footnote{Ver http://en.wikipedia.org/wiki/Backus\_Naur\_Form}, Wirth\footnote{Ver http://en.wikipedia.org/wiki/Wirth\_syntax\_notation} e com diagramas de sintaxe.

Na terceira etapa, implementamos o módulo referente à parte sintática para a nossa linguagem. O analisador sintático construído obtém uma cadeia de \emph{tokens} proveniente do analisador léxico, e verifica se a mesma pode ser gerada pela gramática da linguagem e, com isso, constrói a árvore sintática \cite{alfred1986compilers}.

Para a quarta entrega, focamos no ambiente de execução. O compilador por nós criado terá como linguagem de saída um programa que será executado na máquina virtual conhecida como Máquina de von Neumann (MVN).

Para a entrega atual, buscamos completar a especificação do código gerado pelo compilador e das rotinas do ambiente de execução da nossa linguagem de alto nível, a CZAR.

Como material de consulta, além de sites sobre o assunto e das aulas ministradas, foi utilizado o livro indicado pelo professor no começo das aulas \cite{intro-compiladores}, para pesquisa de conceitos e possíveis implementações.

O documento apresenta a seguir o que foi solicitado na quinta etapa.


\chapter{Instruções da Linguagem de Saída}
\label{chap:instrucoes-saida}
	% !TEX encoding = UTF-8 Unicode

As instruções da MVN podem ser resumidas pela tabela da figura~\ref{fig:instrucoes-mvn}.

\begin{figure}[ht]
	\centering
	\caption{Lista de instruções da MVN}
	\includegraphics[width=\textwidth]{images/instrucoes-mvn.png}
	\label{fig:instrucoes-mvn}
\end{figure}

A seguir, especificaremos o que é realizado pela máquina ao executar cada tipo de operação.		

\begin{itemize}
	\item Registrador de instrução = 0 (desvio incondicional)

	Modifica o conteúdo do registrador de Endereço da Próxima Instrução (IC) armazenando nele o conteúdo do registrador de operando (OI)

	IC := OI

	\item Registrador de instrução = 1 (desvio se acumulador é zero)

	Se o conteúdo do acumulador (AC) for zero, então modifica o conteúdo do registrador de Endereço da Próxima Instrução (IC), armazenando nele o conteúdo do registrador de operando (OI) 

	Se AC = 0 então IC := OI 
	
	Se não IC := IC + 1 

	\item Registrador de instrução = 2 (desvio se negativo)

	Se o conteúdo do acumulador (AC) for negativo, isto é, se o bit mais significativo for 1, então modifica o conteúdo do registrador de Endereço da Próxima Instrução (IC) armazenando nele o conteúdo do registrador de operando (OI)

	Se AC < 0 então IC := OI 
	
	Se não IC := IC + 1

	\item Registrador de instrução = 3 (constante para acumulador)

	Armazena no acumulador (AC) o número relativo de 12 bits contido no registrador de operando (OI), estendendo seu bit mais significativo (bit de sinal) para completar os 16 bits do acumulador
		
	AC := OI 
	
	IC := IC +1 

	\item Registrador de instrução = 4 (soma)

	Soma ao conteúdo do acumulador (AC) o conteúdo da posição de memória indicada pelo registrador de operando MEM[OI]. Guarda o resultado no acumulador

	AC := AC + MEM[OI] 

	IC := IC + 1

	\item Registrador de instrução = 5 (subtração)

	Subtrai do conteúdo do acumulador (AC) o conteúdo da posição de memória indicada pelo registrador de operando MEM[OI]. Guarda o resultado no acumulador

	AC := AC - MEM[OI]

	IC := IC + 1 
		
	\item Registrador de instrução = 6 (multiplicação)

	Multiplica o conteúdo do acumulador (AC) pelo conteúdo da posição de memória indicada pelo registrador de operando MEM[OI]. Guarda o resultado no acumulador

	AC := AC * MEM[OI] 

	IC := IC + 1

	\item Registrador de instrução = 7 (divisão inteira)

	Dividir o conteúdo do acumulador (AC) pelo conteúdo da posição de memória indicada pelo registrador de operando MEM[OI]. Guarda a parte inteira do resultado no acumulador

	AC := int (AC / MEM[OI])

	IC := IC + 1 
			
	\item Registrador de instrução = 8 (memória para acumulador)

	Armazena no acumulador (AC) o conteúdo da posição de memória endereçada pelom registrador de operando (OI) 

	AC := MEM[OI]		

	IC := IC + 1
			
	\item Registrador de instrução = 9 (acumulador para memória)

	Guarda o conteúdo do acumulador (AC) na posição de memória endereçada pelo registrador de operando (OI) 

	MEM[OI] := AC		
	
	IC := IC + 1 
			
	\item Registrador de instrução = A (desvio para subprograma)

	Armazena o conteúdo do registrador de Endereço da Próxima Instrução (IC), incrementado de uma unidade, no registrador de endereço de retorno (RA). Armazena no registrador de Endereço da Próxima Instrução (IC) o conteúdo do registrador de operando (OI).

	RA := IC + 1
	
	IC := OI

	\item Registrador de instrução = B (retorno de subprograma)

	Armazena no registrador de Endereço da Próxima Instrução (IC) o conteúdo do registrador de endereço de retorno (RA), e no acumulador (AC) o conteúdo da posição de memória apontada pelo registrador de operando (OI) 

	AC := MEM[OI]			

	IC := RA 	 	 	 		

			
	\item Registrador de instrução = C (stop)

	Modifica o conteúdo do registrador de Endereço da Próxima Instrução (IC) armazenando nele o conteúdo do registrador de operando (OI) e para o processamento

	IC := OI

	\item Registrador de instrução = D (input)
 					
	Aciona o dispositivo padrão de entrada e aguardar que o usuário forneça o próximo dado a ser lido. Transfere o dado para o acumulador 

	Aguarda
	
	AC := dado de entrada 
	
	IC := IC + 1 
		
	\item Registrador de instrução = E (output)

	Transfere o conteúdo do acumulador (AC) para o dispositivo padrão de saída. Aciona o dispositivo padrão de saída e aguardar que este termine de executar a operação de saída 

	dado de saída := AC 

	aguarda
	
	IC := IC + 1

	\item Registrador de instrução = F (supervisor call)

	Não implementado: por enquanto esta instrução não faz nada.

	IC := IC + 1
\end{itemize}

Escrever um programa usando diretamente codificação binária não é uma tarefa simples, e tampouco agradável. Naturalmente, se um programa é muito grande ou se lida com diversas estruturas complexas (listas, etc.), a sua codificação se torna ainda mais difícil e complexa.

Por conta disso, torna-se imprescindível construir alguma abstração que facilite a programação e a verificação dos programas. A primeira idéia, mais natural, é utilizar o modelo de máquina existente e, a partir dele, definir nomes (mnemônicos) para cada instrução da máquina. Posteriormente, verifica-se que somente isso não basta, pois é necessário lidar com os endereços dentro de um programa (rótulos, operandos, sub-rotinas), com a reserva de espaço para tabelas, com valores constantes. Enfim, é necessário definir uma linguagem simbólica.

\begin{figure}[ht]
	\centering
	\caption{Esquema geral de um montador}
	\includegraphics[width=\textwidth]{images/esquema-montador.png}
	\label{fig:esquema-montador}
\end{figure}

Para a construção de um montador, cujo esquema geral está representado na figura~\ref{fig:esquema-montador} pressupõe-se que sejam tratadas as seguintes questões:

\begin{itemize}
	\item definição das instruções: determinar os mnemônicos que as representam simbolicamente;
	\item definição das pseudo-instruções: determinar os mnemônicos que as representam, bem como sua função para o montador.
\end{itemize}

As instruções para a MVN são apresentadas na figura~\ref{fig:mnemonicos-mvn}.

\begin{figure}[ht]
	\centering
	\caption{Tabela de mnemônicos para a MVN (de 2 caracteres)}
	\includegraphics[width=\textwidth]{images/mnemonicos-mvn.png}
	\label{fig:mnemonicos-mvn}
\end{figure}


\chapter{Pseudoinstruções da Linguagem de Saída}
\label{chap:pseudoinstrucoes-saida}
	% !TEX encoding = UTF-8 Unicode

TODO Listar e descrever pseudoinstruções.


\chapter{Características Gerais}
\label{chap:caracteristicas-gerais}
	% !TEX encoding = UTF-8 Unicode

TODO Descrever características gerais do ambiente de execução. Por exemplo: organização da memória, acesso a variáveis, chamadas de subrotinas.

	
\chapter{Biblioteca Desenvolvida em Assembly}
\label{chap:biblioteca-assembly}
	% !TEX encoding = UTF-8 Unicode

A biblioteca padrão desenvolvida é dividida em dois módulos o primeiro
implementa as funções básicas de empilhamento, sendo chamado de \verb!std.asm!.
O segundo módulo implementa as operações de input e output de dados, de nome 
\verb!stdio.asm!.

\section{STD}

A manipulação de pilhas é feita pela biblioteca padrão, sendo que deseja-se
seguir a estrutura abaixo definida, facilitando o uso e acesso das variáveis.
Vemos abaixo um exemplo do uso da biblioteca. Na linha 23 salvamos uma variável
recebida por parâmetro na pilha e na linha 30 recuperamos seu valor. 

Logo antes de retornar devemos executar a função \verb!POP_CALL! 
ela é responsável por
escrever o endereço de retorno na função em que estamos, assim aproveitando das
chamadas existentes na \verb!MVN! (funções devem ser \emph{stateless} para tanto). Percebe-se
que é possível executar a função recursivamente (linha 55 e 57). Para tanto é
necessário chamar a função \verb!PUSH_CALL! para que a mesma efetue o
empilhamento e escreva o endereço de retorno atual na pilha. 

\begin{lstlisting}[basicstyle=\footnotesize,numbers=left,breaklines=true,morekeywords={}]
;; VARIAVEIS GLOBAIS
;; comeco da pilha = FFF 
;; tamanho da pilha = 2FF 
;;   | ptr to old_stack_head  | \___ STACK_PTR
;;   |      savedregist       |   
;;   |         ...            |   
;;   |      local var         |  
;;   |         ...            |   
;;   |      temporaries       |  
;;   |      parameters        |   
;;   |         ...            |  
;;   |     ref parameters     |  ____ OLD STACK_PTR
;;   |      returnaddrs       | /     (STACK_PTR points here)

EXAMPLE_STACK_ARG    K  /0000
EXAMPLE_STACK        JP /000 
                     SC PRINT_STACK_ADDRS   ;; deve imprimir 0fff 
                     ;;; SALVAR ARGUMENTOS na pilha 
                     LV =0
                     MM WORD_TO_SAVE 
                     LV EXAMPLE_STACK_ARG 
                     MM ORIGIN_PTR 
                     SC SAVE_WORD_TO_LOCAL_VAR 
                     ;;;; CORPO DA FUNCAO 
                     ;;; CARREGANDO UM VALOR DA PILHA 
                     LV =0
                     MM WORD_TO_GET 
                     LV EXAMPLE_STACK_ARG 
                     MM STORE_PTR
                     SC GET_WORD_LOCAL_VAR 
                     ;;; IMPRIME 
                     LV COUNT_IS
                     MM STRING_PTR 
                     SC P_STRING  ;; inline fct, no need to stack 
                     LD EXAMPLE_STACK_ARG 
                     MM TO_BE_PRINTED 
                     SC P_INT_ZERO
                     SC P_LINE

                     LD EXAMPLE_STACK_ARG
                     JZ RETURN_EXAMPLE_STACK

                     LD EXAMPLE_STACK_ARG 
                     -  ONE 
                     MM EXAMPLE_STACK_ARG 

                     LV =1
                     MM PUSH_CALL_SIZELV
                     LV =0
                     MM PUSH_CALL_RET_ADDRS 
                     LV =0
                     MM PUSH_CALL_TMP_SZ
                     LV =0
                     MM PUSH_CALL_PAR_SZ 
                     SC PUSH_CALL

                     SC EXAMPLE_STACK  ;; chamada recursiva
                     ;;;; FIM DO CORPO DA FUNCAO 
RETURN_EXAMPLE_STACK LV EXAMPLE_STACK
                     MM POP_CALL_FCT 
                     SC POP_CALL ;; trickery!

                     SC PRINT_STACK_ADDRS   ;; deve imprimir 0fff 
                     RS EXAMPLE_STACK
\end{lstlisting}

Abaixo podemos ver a implementação das funções de \verb!PUSH! e \verb!POP!

A pilha é implementada dos valores mais altos da memória para os valores mais
baixos, sendo assim, o ponteiro de pilha começa apontando para \verb!0x0FFF!.

A pilha funciona como uma lista ligada que guarda o endereço da última célula
da pilha. Sendo assim, a operação de \verb!POP! é trivial. Estas funções fazem
a gestão do endereço de retorno automaticamente, contanto que se siga a
premissa de chamada (chamada da função logo após a chamada de \verb!PUSH_CALL!
e seus parâmetros).

\begin{lstlisting}[basicstyle=\footnotesize,numbers=left,breaklines=true,morekeywords={}]
;; *** PUSH_CALL ***
PUSH_CALL           JP /000 
                    LD PUSH_CALL  ;; get return addrs 
                    +  TWO ;; return address of the callee
                    +  LOADV_CONST
                    MM LOAD_RETURN_ADDRS 
                    LD STACK_PTR
                    -  TWO        ;; new return addrs  
                    +  MOVE_CONST 
                    MM MOVE_RETURN_ADDRS 
LOAD_RETURN_ADDRS   JP /000  
MOVE_RETURN_ADDRS   JP /000   ;; return addrs salvo 
                    LD STACK_PTR          
                    -  TWO 
                    -  TWO 
                    -  PUSH_CALL_SIZELV
                    -  PUSH_CALL_RET_ADDRS
                    -  PUSH_CALL_TMP_SZ
                    -  PUSH_CALL_PAR_SZ 
                    -  TWO  ;; return addrs
                    MM TMP_1
                    LD TMP_1 
                    +  MOVE_CONST 
                    MM MRKR_PC_SAVE_HEAD 
                    LD STACK_PTR 
MRKR_PC_SAVE_HEAD   JP /000 
                    LD TMP_1 
                    MM STACK_PTR
                    RS PUSH_CALL
;; **** POP_CALL ****

POP_CALL_FCT        K /0000             
POP_CALL            JP /000 ; retorno 
POP_CALL_INIT       LD STACK_PTR 
                    +  LOAD_CONST 
                    MM MRKR_PC_LOAD_HEAD 
MRKR_PC_LOAD_HEAD   JP /000 
                    MM STACK_PTR 
                    LD STACK_PTR 
                    -  TWO
                    +  LOAD_CONST 
                    MM LOAD_RETURN_ADDRS_2
                    LD POP_CALL_FCT 
                    +  MOVE_CONST 
                    MM MOVE_RETURN_ADDRS_2
LOAD_RETURN_ADDRS_2 JP /000 
MOVE_RETURN_ADDRS_2 JP /000  ;; engana a funcao para ela pensar que ela 
                             ;; tem que retornar para esse valor 
                    RS POP_CALL
\end{lstlisting}


As rotinas de salvaguarda e carregamento dos valores locais, parâmetros,
   referências pode ser feita por meio das chamadas abaixo, 
   \verb!SAVE_WORD_TO_LOCAL_VAR! e \verb!GET_WORD_LOCAL_VAR! respectivamente.

\begin{lstlisting}[basicstyle=\footnotesize,numbers=left,breaklines=true,morekeywords={}]
;; **** SAVE_WORD_TO_LOCAL_VAR WORD_TO_SAVE ORIGIN_PTR ****
SAVE_WORD_TO_LOCAL_VAR      JP /000 
                            LD STACK_PTR
                            + TWO          ;; first word 
                            + WORD_TO_SAVE 
                            + WORD_TO_SAVE  ;; WORD_TO_GET * 2
                            + MOVE_CONST   ;; 
                            MM MOVE_WORD_LOCAL_VAR_2
                            LD ORIGIN_PTR
                            + LOAD_CONST 
                            MM LOAD_WORD_LOCAL_VAR_2
LOAD_WORD_LOCAL_VAR_2       JP /000 ;; 8FROMPTR
MOVE_WORD_LOCAL_VAR_2       JP /000 ;; 9TOPTR
                            RS SAVE_WORD_TO_LOCAL_VAR

;; **** GET_WORD_LOCAL_VAR WORD_TO_GET STORE_PTR ****


WORD_TO_GET        K /000 
STORE_PTR          K /000

GET_WORD_LOCAL_VAR          JP /000 
                            LD STACK_PTR
                            + TWO          ;; first word 
                            + WORD_TO_GET  
                            + WORD_TO_GET  ;; WORD_TO_GET * 2
                            + LOAD_CONST   ;; 
                            MM LOAD_WORD_LOCAL_VAR  
                            LD STORE_PTR
                            + MOVE_CONST 
                            MM MOVE_WORD_LOCAL_VAR
LOAD_WORD_LOCAL_VAR         JP /000 ;; 8FROMPTR
MOVE_WORD_LOCAL_VAR         JP /000 ;; 9TOPTR
                            RS GET_WORD_LOCAL_VAR
 \end{lstlisting}


 \section{STDIO}


O ambiente de execução também é provido de funções de input/output:

Para a impressão de \emph{strings} podemos utilizar a função \verb!P_STRING!, 
     passando o ponteiro para o começo de uma \verb!string!. Em \verb!CZAR!
     consideramos \emph{strings} como sendo \emph{bytes} em um vetor de
     \emph{word} terminados pelo \emph{byte} \verb!0x0000!. Vale salientar que
     esta forma de armazenamento não causa problemas com outros tipos de
     armazenamento mais compactos, como a utilização dos dois \emph{bytes}
     da \emph{word} para armazenamento de \emph{chars} subsequentes. Quando a
     função recebe a \emph{word} \verb!0x0030!, primeiramente ela vai imprimir
     \verb!0x00! que é o caractere nulo, portanto, sem impressão e então
     imprimir o caractere correspondente a \verb!0x30!. 

\begin{lstlisting}[basicstyle=\footnotesize,numbers=left,breaklines=true,morekeywords={}]
;; ****  P_STRING &STRING_PTR ****
;;   Imprime a string apontada por STRING_PTR ate
;; o caractere /000  

P_STRING            JP /000           ; endereco de retorno 
PSTRINGINIT         LD STRING_PTR 
                    MM TO_BE_PRINTED_TMP 
LOAD_TO_BE_PRINTED  LD TO_BE_PRINTED_TMP
                    +  LOAD_CONST    
                    MM LABELLOAD 
LABELLOAD           K  /0000 
                    JZ P_STRING_END  ; se zero vamos para o final!
                    PD /100 
                    LD TO_BE_PRINTED_TMP
                    +  TWO
                    MM TO_BE_PRINTED_TMP
                    JP LOAD_TO_BE_PRINTED
P_STRING_END        RS P_STRING 

 \end{lstlisting}

Para a leitura de \emph{strings} seguimos o padrão definido anteriormente, um
\emph{byte} (\emph{char}) por \emph{word}:

\begin{lstlisting}[basicstyle=\footnotesize,numbers=left,breaklines=true,morekeywords={}]
;; *** GETS STORE_PTR_IO ***
;; Existe um problema de buffer aqui... nao vamos 
;; trata-lo, pois este e' um problema intri'nseco da 
;; MVN. (leitura e subsequente bloqueio por word)
LAST_CONTROL_CHAR_P_ONE    K /0021
ARRAY_POS_BYTE  JP /000
GETS            JP /000
                LD STORE_PTR_IO
                MM ARRAY_POS_BYTE
GETS_LOOP       GD /000
                MM HIGH_V
                SC HIGH_LOW 
                LD HIGH_V 
                -  LAST_CONTROL_CHAR_P_ONE 
                JN RETURN_GETS 
                LD ARRAY_POS_BYTE 
                +  MOVE_CONST 
                MM MOVE_HIGH_V
                LD HIGH_V 
MOVE_HIGH_V     JP /000 

                LD ARRAY_POS_BYTE
                +  TWO 
                MM ARRAY_POS_BYTE 

                LD LOW_V 
                -  LAST_CONTROL_CHAR_P_ONE 
                JN RETURN_GETS 
                LD ARRAY_POS_BYTE 
                +  MOVE_CONST 
                MM MOVE_LOW_V
                LD LOW_V 
MOVE_LOW_V      JP /000 

                LD ARRAY_POS_BYTE
                +  TWO 
                MM ARRAY_POS_BYTE 

                JP GETS_LOOP 

RETURN_GETS     LD ARRAY_POS_BYTE 
                +  MOVE_CONST 
                MM MOVE_ZERO
                LV =000  
MOVE_ZERO       JP /000 

                LD ARRAY_POS_BYTE
                +  TWO 
                MM ARRAY_POS_BYTE 
                RS GETS

\end{lstlisting}

A biblioteca também é capaz de realizar a leitura e escrita de valores
inteiros (funções auxiliares estão disponíveis no pacote em anexo): 

\begin{lstlisting}[basicstyle=\footnotesize,numbers=left,breaklines=true,morekeywords={}]
;; *** READ_INT STORE_PTR_IO ***
;; doesnt care about buffers, should have a trailing char at the end of the
;; stream otherwise it will just discard it.. 
STORE_PTR_IO        JP /000
ZERO_M_ONE          K  /002F
NINE_P_ONE          K  /0039

LOW                 K  /0000
HIGH                K  /0000
GO_IF_NUMBER        K  /0000 
TO_BE_TRIMMED       K  /0000 
TBT_TMP             K  /0000 

TRIM_INT            JP /000
                    LD TO_BE_TRIMMED
                    /  SHIFT_BYTE  
                    *  SHIFT_BYTE
                    MM TBT_TMP 
                    LD TO_BE_TRIMMED
                    -  TBT_TMP 
                    MM TO_BE_TRIMMED 
                    RS TRIM_INT 

READ_INT_WORD       JP /000
                    GD /000 
                    MM TMP_3 
                    LD TMP_3
                    /  SHIFT_BYTE
                    MM TO_BE_TRIMMED 
                    SC TRIM_INT 
                    LD TO_BE_TRIMMED
                    MM HIGH 

                    LD TMP_3
                    MM TO_BE_TRIMMED 
                    SC TRIM_INT
                    LD TO_BE_TRIMMED
                    MM LOW
                    RS READ_INT_WORD

READ_INT            JP /000 
                    LV =0 
                    MM TMP_4
READ_INT_LOOP       SC READ_INT_WORD 
                    LD HIGH 
                    MM TMP_3 
                    LV CONT1
                    MM GO_IF_NUMBER
                    JP IF_NUMBER_CONTINUE 
CONT1               LD LOW
                    MM TMP_3 
                    LV READ_INT_LOOP
                    MM GO_IF_NUMBER
                    JP IF_NUMBER_CONTINUE 
NOT_NUMBER          LD STORE_PTR_IO
                    +  MOVE_CONST 
                    MM MOVE_READ_INT
                    LD TMP_4 
MOVE_READ_INT       JP /000
                    RS READ_INT 

IF_NUMBER_CONTINUE  LD TMP_3 
                    -  ZERO_M_ONE 
                    JN NOT_NUMBER 
                    LD NINE_P_ONE 
                    -  TMP_3
                    JN NOT_NUMBER  

                    LD TMP_4
                    *  TEN 
                    MM TMP_4 


                    LD TMP_3 
                    -  ZERO_M_ONE
                    -  ONE 
                    +  TMP_4
                    MM TMP_4

                    LD GO_IF_NUMBER
                    MM END_READ_INT
END_READ_INT        JP /000
\end{lstlisting}

A impressão de inteiros, por ser crítica e muito importante para a correção de
erros, foi feita de forma simples e direta. Sem laços (unwind de \verb!GOTO!
    explícito) ou complicações,
    resultando em uma função bem determinada e robusta.
\begin{lstlisting}[basicstyle=\footnotesize,numbers=left,breaklines=true,morekeywords={}]
;; *** P_INT_ZERO TO_BE_PRINTED ***
;;  Imprime um inteiro (com zeros a esquerda)
;; ex:  
;;  INT_2 K  =345 
;;        LD INT_2
;;        MM TO_BE_PRINTED 
;;        SC P_INT_ZERO
;; imprime 00345
;;
;;
;; Esta funcao esta com o loop inline
;; sendo simples e robusta

P_INT_ZERO          JP /000
P_INT_INIT          JP P_INT_REAL_INIT
ZERO_BASE           K /30
;; bases para a conversao:
INT_POT_1           K =10000   
INT_POT_2           K =1000 
INT_POT_3           K =100  
INT_POT_4           K =10 
INT_POT_5           K =1 
P_INT_REAL_INIT     LD TO_BE_PRINTED       ;; PRIMEIRO CHAR
                    MM TMP_1 
                    /  INT_POT_1 
                    +  ZERO_BASE 
                    PD /100                     ;; imprime 
                    LD TMP_1  
                    /  INT_POT_1
                    *  INT_POT_1 
                    MM TMP_2                     
                    LD TMP_1
                    -  TMP_2
                    MM TMP_1   
                    /  INT_POT_2                ;; segundo char 
                    +  ZERO_BASE 
                    PD /100                     ;; imprime 
                    LD TMP_1  
                    /  INT_POT_2
                    *  INT_POT_2 
                    MM TMP_2                     
                    LD TMP_1
                    -  TMP_2
                    MM TMP_1 
                    /  INT_POT_3                ;; terceiro char 
                    +  ZERO_BASE 
                    PD /100                     ;; imprime 
                    LD TMP_1  
                    /  INT_POT_3
                    *  INT_POT_3 
                    MM TMP_2                     
                    LD TMP_1
                    -  TMP_2
                    MM TMP_1 
                    /  INT_POT_4                ;; quarto char 
                    +  ZERO_BASE 
                    PD /100                     ;; imprime 
                    LD TMP_1  
                    /  INT_POT_4
                    *  INT_POT_4 
                    MM TMP_2                     
                    LD TMP_1
                    -  TMP_2
                    MM TMP_1 
                    /  INT_POT_5                ;; quinto char 
                    +  ZERO_BASE 
                    PD /100                     ;; imprime 
                    LD TMP_1  
                    /  INT_POT_5
                    *  INT_POT_5 
                    MM TMP_2                     
                    LD TMP_1
                    -  TMP_2
                    MM TMP_1 
                    RS P_INT_ZERO

 \end{lstlisting}

 \chapter{Exemplo de Execução}

 Foi escrito um script em \verb!BASH! para facilitar a compilação e execução da
 MVN:

\begin{lstlisting}[basicstyle=\footnotesize,numbers=left,breaklines=true,morekeywords={}]
# mvnrc 
# 
# Author: gpg
#
# Usage:
#  $ source ./mvnfunctions.sh 
#  $ makelib libraries.asm
#  $ makemain mainfile.asm
#    be happy :D

JAVARUN="java -cp"
MVNDL="java -jar MvnPcs4_wDumperLoader.jar"
MLR="$JAVARUN PCS2302_MLR.jar"
MONTA="$MLR montador.MvnAsm"
LINKA="$MLR linker.MvnLinker"
RELOCA="$MLR relocator.MvnRelocator"

MVNFILES=mvnfiles 
# Usage:
# makelib lib.asm 
function makelib () {
    $MONTA $1 && {
        echo `basename $1 .asm`.mvn >> $MVNFILES
        return 0
    }
    return 1
}

function prog_size() {
    lines=$(cat $1 | wc -l)
    echo $lines*2 | bc 1>&2
    echo $lines*2+100 | bc
}

function clean_reloca() {
    rm -f relocados.list
}

function reloca_var() {
    OUTNAME=`basename $1 .mvn`_relocado.mvn
    OUTSIZE=$(prog_size $1)
    echo RELOCANDO $1, comeco: $2, tamanho: $OUTSIZE 1>&2 
    START=$(printf "%X\n" $2)
    $RELOCA $1 $OUTNAME $2 1>&2 || return 1
    cat $OUTNAME 1>&2
    echo $OUTNAME >> relocados.list
    echo $OUTSIZE | bc
}

function makemain () {
    OUTNAME=`basename $1 .asm`.out 
    MAIN=`basename $1 .asm`.mvn 
    BINARY=`basename $1 .asm`

    TMPNAME=tmpfile.tmp

    # make main or die 
    $MONTA $1 || {
        return 1
    }
    {
        clean_reloca 

        SIZEREL=$(prog_size $MAIN) || return 1
        BUTTER=$(cat $MVNFILES | sort -u)
        echo Ligando as bagaca sizerel is $SIZEREL
        $LINKA $MAIN $BUTTER -s $OUTNAME && {
            echo Jogando tudo pra baixo da main 
            $RELOCA $OUTNAME $BINARY $SIZEREL && {
                echo "All Ok!"
                echo -e '\E[37;44m'"\033[1mYour binary is called: $BINARY\033[0m"
                borala
            } || {
                echo Error 
                return 0
            }
        } || {
            echo Error
            return 0
        }
    }
}

function cleanlib () {
    rm -f $MVNFILES
}

function clean_all_mvn () {
    rm -f *.dump
    rm -f *.lst
    rm -f *.mvn
}

function borala () {
    rlwrap $MVNDL
}
\end{lstlisting}


\section{Exemplo de chamada recursiva}

Temos aqui um exemplo de leitura e chamada recursiva:

O programa pede por um número na linha 76, uma vez digitado (deve-se utilizar a
    tecla ENTER duas vezes devido ao problema de \emph{buffer} citado no
    capítulo anterior) o programa carrega o valor em uma variável local e passa
a mesma para uma função que reduz o valor indicado por uma unidade e chama ela
mesma até que o valor se resume a zero, neste momento a função retorna e toda a
pilha é desalocada. 



\begin{lstlisting}[basicstyle=\footnotesize,numbers=left,breaklines=true,morekeywords={}]

gpg@rancheiro: --( ~/Poli/PCS2056-Compiladores/github/compiladores/MVN )
$  . mvnfunctions.sh 
gpg@rancheiro: --( ~/Poli/PCS2056-Compiladores/github/compiladores/MVN )
$  makelib std.asm 
===============================================================================
              PCS2302/PCS2024  Montador da Maquina de Von Neumann
                 Versao 1.1 (a)2010  Todos os direitos reservados

Montador finalizou corretamente, arquivos gerados.
gpg@rancheiro: --( ~/Poli/PCS2056-Compiladores/github/compiladores/MVN )
$  makelib stdio.asm 
===============================================================================
              PCS2302/PCS2024  Montador da Maquina de Von Neumann
                 Versao 1.1 (a)2010  Todos os direitos reservados

Montador finalizou corretamente, arquivos gerados.
gpg@rancheiro: --( ~/Poli/PCS2056-Compiladores/github/compiladores/MVN )
$  makemain usestd.asm
===============================================================================
              PCS2302/PCS2024  Montador da Maquina de Von Neumann
                 Versao 1.1 (a)2010  Todos os direitos reservados

Montador finalizou corretamente, arquivos gerados.
352
Ligando as bagaca sizerel is 452
Arquivo gerado com sucesso.
Jogando tudo pra baixo da main
Arquivo gerado com sucesso.
All Ok!
Your binary is called: usestd
Inicializacao padrao de dispositivos
MVN Inicializada

                Escola Politecnica da Universidade de Sao Paulo
             PCS2302/PCS2024 - Simulador da Maquina de von Neumann
         MVN versao 4.2 (Novembro/2010) - Todos os direitos reservados

 COMANDO  PARAMETROS           OPERACAO
---------------------------------------------------------------------------
    i                          Re-inicializa MVN
    p     [arq]                Carrega programa para a memoria
    r     [addr] [regs]        Executa programa
    b                          Ativa/Desativa modo Debug
    l                          Loader
    d                          Dumper
    s                          Manipula dispositivos de I/O
    g                          Lista conteudo dos registradores
    m     [ini] [fim] [arq]    Lista conteudo da memoria
    h                          Ajuda
    x                          Finaliza MVN e terminal

> p usestd
Programa usestd carregado

> r 000
Exibir valores dos registradores a cada passo do ciclo FDE (s/n)[s]: n
please insert a string:
Poli_USP!!!
Poli_USP!!!!"#Teste$

00000
00001
00002
00003
00004
00005
00006
00007
00008
00009
00010
00011

Stack should be:04095
Please write a number, we will count recursively until it gets to zero: 
23

Stack was: 04088
Counter is:00023
Stack was: 04081
Counter is:00022
Stack was: 04074
Counter is:00021
Stack was: 04067
Counter is:00020
Stack was: 04060
Counter is:00019
Stack was: 04053
Counter is:00018
Stack was: 04046
Counter is:00017
Stack was: 04039
Counter is:00016
Stack was: 04032
Counter is:00015
Stack was: 04025
Counter is:00014
Stack was: 04018
Counter is:00013
Stack was: 04011
Counter is:00012
Stack was: 04004
Counter is:00011
Stack was: 03997
Counter is:00010
Stack was: 03990
Counter is:00009
Stack was: 03983
Counter is:00008
Stack was: 03976
Counter is:00007
Stack was: 03969
Counter is:00006
Stack was: 03962
Counter is:00005
Stack was: 03955
Counter is:00004
Stack was: 03948
Counter is:00003
Stack was: 03941
Counter is:00002
Stack was: 03934
Counter is:00001
Stack was: 03927
Counter is:00000
Stack was: 03934
Stack was: 03941
Stack was: 03948
Stack was: 03955
Stack was: 03962
Stack was: 03969
Stack was: 03976
Stack was: 03983
Stack was: 03990
Stack was: 03997
Stack was: 04004
Stack was: 04011
Stack was: 04018
Stack was: 04025
Stack was: 04032
Stack was: 04039
Stack was: 04046
Stack was: 04053
Stack was: 04060
Stack was: 04067
Stack was: 04074
Stack was: 04081
Stack was: 04088
Stack was: 04095
Stack should be:04095

\end{lstlisting}



%\bibliography{bibliografia}

\end{document}
