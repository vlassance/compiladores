% !TEX encoding = UTF-8 Unicode

\section{Organização da memória}

O ambiente de execução da MVN fornece aos programadores um total de 4Kb de memória para ser usado tanto para o código quanto para as variáveis do programa. Desses 4Kb, a parte inicial da memória é reservada para guardar as instruções que serão executadas pelo programa. A parte final da memória deve ser usada especialmente para o uso do registro de ativação.

\section{Registro de ativação}

O registro de ativação nesse ambiente de execução será feito sob a forma de uma pilha, onde a cada chamada de função todos os dados do referentes a função, bem como o endereço de retorno, devem ser empilhados para serem usados. Os dados a serem empilhados no registro de ativação são:

- Endereço de retorno
- Endereço do próximo endereço da pilha
- Parâmetros da função
- Variáveis locais das função

O endereço de retorno fica localizado no primeiro endereço do bloco empilhado no registro de ativação. O segundo endereço é referente ao endereço do primeiro endereço do próximo bloco do registro de ativação. Esse endereço é usado para mudar o valor ponteiro do registro de ativação, para que a função que chamou a outra possa voltar a enxergar suas variáveis. Do terceiro endereço em diante estão localizados os parâmetros da função. Após o final dos parâmetros, estão localizadas as variáveis locais necessárias para guardar executar as operações durante a execução da função.

Os endereços das variáveis locais e dos parâmetros podem ser calculados usando o ponteiro do registro de ativação, somando dois mais os tamanhos das variáveis existentes anteriormente.

A figura abaixo ilustra a organização da pilha de ativação.

registros-ativacao.png

O uso do registro de ativação permite entre outras coisas a chamada recursiva de funções, uma vez isso não é possível de forma nativa no ambiente da MVN. Com registro de ativação, realizar uma recursão significa empilhar um novo bloco à pilha e relançar a execução da função.
