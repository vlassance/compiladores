% !TEX encoding = UTF-8 Unicode

\textbf{Explique como enriquecer esse analisador léxico com um expansor de macros do tipo \#DEFINE, não paramétrico nem recursivo, mas que permita a qualquer macro chamar outras macros, de forma não cíclica.}

Uma forma de permitir a utilização de macros seria o de realizar o pré-processamento, substituindo no código todas as macros pelos seus valores. Caso isso não seja desejado, também é possível acrescentar o tratamento de macros no analisador léxico, como explicitado na questão.

A maneira mais prática seria de se armazenar todas as macros em um vetor ou \emph{hash table}, de forma similar a como é feito com as palavras reservadas. Ao se encontrar um identificador de macro, deve-se adicionar a um buffer o conteúdo da macro e deve-se processar o buffer até que o mesmo termine, antes de retornar a leitura do arquivo. Caso haja um identificador de macro dentro da definição de uma macro, pode-se substituir o identificador pela sua definição dentro do buffer que já está sendo lido, pra facilitar a lógica.

Uma possível solução para não tratar macros como casos únicos de utilização de buffers de leitura, pode-se ler um conjunto de caracteres a cada vez do arquivo e sempre armazenar em um buffer, lendo o arquivo novamente somente quando o buffer estiver vazio. Dessa forma, a macro nada mais será que uma substituição de um identificador por uma definição.
