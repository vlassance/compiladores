% !TEX encoding = UTF-8 Unicode

\textbf{Relate detalhadamente o funcionamento do analisador léxico assim construído, incluindo no relatório: descrição teórica do programa; descrição da sua estrutura; descrição de seu funcionamento; descrição dos testes realizados e das saídas obtidas.}

O analisador léxico lê um arquivo de configuração da máquina de estados
(transdutor). O mesmo pode ser comparado à seguinte regex:

\verb!(.)([^\1]*)\1\s*([A-Za-z]+(:?_[0-9]+)?)\s*([A-Za-z]+(:?_[0-9]+)?)!

Cada linha possui uma lista de caracteres delimitados por um caractere
especial (por exemplo `+' ou `\verb!#!') e dois identificadores que designam os
estados inicial e final da transição. O caractere \verb!@! designa todas as
transições, este é usado principalmente para encaminhar qualquer aceitação
final de um sub-autômato ao estado \verb!Q0!, para então ser tratado
normalmente.

Ex:
\begin{lstlisting}
+abcdefghijklmnopqrstuvxywz+    Q0      IDENT
+ABCDEFGHIJKLMNOPQRSTUVXYWZ+    Q0      IDENT
+_+                             Q0      IDENT
\end{lstlisting}
Este exemplo vai reconhecer todos os caracteres a-z e A-Z mais o underscore
(\_) como transições do estado \verb!Q0! e \verb!IDENT!.

Após a leitura do arquivo de configuração e de um arquivo com
    \emph{keywords}, faz-se a leitura do arquivo fonte, por meio do
transdutor, percorrendo-se o arquivo fonte token a token. Uma função
\verb!next_useful_token! oferece o não retorno dos tokens de espaço, tal qual a
quebra de linha e espaços normais, além de ignorar comentários. O motor do lex também substitui classes
\verb!IDENT! para \verb!RESERVED! se a palavra se encontra na lista de
identificadores reservados. 

Uma \verb+hashtable+ está sendo desenvolvida e será integrada nas próximas versões do
compilador.
