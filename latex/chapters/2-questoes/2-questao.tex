% !TEX encoding = UTF-8 Unicode

\textbf{Quais as vantagens e desvantagens da implementação do analisador léxico como uma fase separada do processamento da linguagem de programação em relação à sua implementação como sub-rotina que vai extraindo um átomo a cada chamada?}

Geralmente, o gargalo encontrado durante a compilação de um programa sem
otimização é a leitura de arquivos e a análise léxica. Separando-se o
analisador léxico do resto do compilador, é possivel otimizar esse módulo e
obter um analisador léxico genérico que serviria a princípio para qualquer
linguagem.

A desvantagem de se separar os dois é o desacoplamento da lógica e, por
conseguinte, das informações disponíveis ao analizador sintático e semântico,
informações estas que podem ser importantes no reconhecimento das classes dos
tokens encontrados dependendo da linguagem a ser compilada.

Exemplo: Shell Script - O primeiro \verb!echo! refere-se ao comando echo e o
segundo refere-se ao primeiro argumento do comando.

\begin{lstlisting}[frame=single,language=bash,numbers=left]
    echo echo
\end{lstlisting}
