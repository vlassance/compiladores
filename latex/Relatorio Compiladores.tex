% verso e anverso:
% \documentclass[12pt,openright,twoside,a4paper,english]{abntex2}
% apenas verso:	
\documentclass[12pt,oneside,a4paper,english]{abntex2} 

\usepackage{cmap}				% Mapear caracteres especiais no PDF
\usepackage{lmodern}			% Usa a fonte Latin Modern			
\usepackage[T1]{fontenc}		% Selecao de codigos de fonte.
\usepackage[utf8]{inputenc}		% Codificacao do documento (conversão automática dos acentos)
\usepackage{lastpage}			% Usado pela Ficha catalográfica
\usepackage{indentfirst}		% Indenta o primeiro parágrafo de cada seção.
\usepackage{color}				% Controle das cores
\usepackage{graphicx}			% Inclusão de gráficos
\usepackage{pdfpages}
\usepackage[brazilian,hyperpageref]{backref}	 % Paginas com as citações na bibl
\usepackage[alf]{abntex2cite}	% Citações padrão ABNT

% Configurações do pacote backref
% Usado sem a opção hyperpageref de backref
\renewcommand{\backrefpagesname}{Citado na(s) página(s):~}
% Texto padrão antes do número das páginas
\renewcommand{\backref}{}
% Define os textos da citação
\renewcommand*{\backrefalt}[4]{
	\ifcase #1 %
		Nenhuma citação no texto.%
	\or
		Citado na página #2.%
	\else
		Citado #1 vezes nas páginas #2.%
	\fi}%

\definecolor{blue}{RGB}{41,5,195} % alterando o aspecto da cor azul

\makeatletter
\hypersetup{
     	%pagebackref=true,
		pdftitle={\@title}, 
		pdfauthor={\@author},
    	pdfsubject={\@title},
	    pdfcreator={\imprimirpreambulo},
		pdfkeywords={Linguagens}{Compiladores}{Analisador Léxico}{Gerenciamento de Desastres}{Sistemas Multiagentes}, 
		colorlinks=true,       		% false: boxed links; true: colored links
    	linkcolor=blue,          	% color of internal links
    	citecolor=blue,        		% color of links to bibliography
    	filecolor=magenta,      		% color of file links
		urlcolor=blue,
		bookmarksdepth=4
}
\makeatother

\autor{Gustavo Pacianotto Gouveia, Victor Lassance}
\title{Relatório de Compiladores - Primeira Etapa - Construção de um analisador léxico}
\orientador[Professor:]{Ricardo Luis de Azevedo da Rocha}
\preambulo{Texto apresentado à Escola Politécnica da Universidade de São Paulo como requisito para a aprovação na disciplina Linguagens e Compiladores no quinto módulo acadêmico do curso de graduação em Engenharia de Computação, junto ao Departamento de Engenharia de Computação e Sistemas Digitais (PCS).}
\instituicao{%
	Universidade de São Paulo
	\par
	Escola Politécnica
	\par
	Engenharia de Computação - Curso Cooperativo}
\local{São Paulo}
\data{2013}
\tipotrabalho{PCS2056 - Linguagens e Compiladores}

\setlength{\parindent}{1.3cm} % O tamanho do parágrafo
\setlength{\parskip}{0.2cm}  % Controle do espaçamento entre um parágrafo e outro

\makeindex

\begin{document}

\frenchspacing % Retira espaço extra obsoleto entre as frases.

\imprimircapa
\imprimirfolhaderosto

\clearpage
\begin{resumo}
	% !TEX encoding = UTF-8 Unicode
Este trabalho descreve a concepção e o desenvolvimento de um compilador utilizando a linguagem C. O escopo do compilador se limita a casos mais simples, porém simbólicos, e que servem ao aprendizado do processo de criação e teste de um compilador completo. A estrutura da linguagem escolhida para ser implementada se assemelha a própria estrutura do C, por facilidade de compreensão, porém com algumas peculiaridades trazidas de outras linguagens.

\vspace{\onelineskip}
    
\noindent
\textbf{Palavras-chaves}: Linguagens, Compiladores, Implementação do Reconhecedor Sintático.

\end{resumo}

\tableofcontents

\textual

\chapter{Introdução}
\label{chap:introducao}
	% !TEX encoding = UTF-8 Unicode

Este projeto tem como objetivo a construção de um compilador de um só passo, dirigido por sintaxe, com analisador e reconhecedor sintático baseado em autômato de pilha estruturado.

Em um primeiro momento, foi definida uma linguagem de programação e identificados os tipos de átomos. Para cada átomo foi escrito uma gramática linear representativa da sua lei de formação e um reconhecedor para o átomo. Desse modo, as gramáticas assim escritas foram unidas e convertidas em um autômato finito, o qual foi transformado em um transdutor e implementado como sub-rotina, dando origem ao analisador léxico propriamente dito. Também foi criada uma função principal para chamar o analisador léxico e possibilitar o seu teste.

Durante a segunda etapa, a sintaxe da linguagem, denonimada por nós de CZAR, foi definida formalmente a partir de uma definição informal e de exemplos de programas que criamos, misturando palavras-chave e conceitos de diferentes linguagens de programação. As três principais definições foram escritas na notação BNF\footnote{Ver http://en.wikipedia.org/wiki/Backus\_Naur\_Form}, Wirth\footnote{Ver http://en.wikipedia.org/wiki/Wirth\_syntax\_notation} e com diagramas de sintaxe.

Na terceira etapa, implementamos o módulo referente à parte sintática para a nossa linguagem. O analisador sintático construído obtém uma cadeia de \emph{tokens} proveniente do analisador léxico, e verifica se a mesma pode ser gerada pela gramática da linguagem e, com isso, constrói a árvore sintática \cite{alfred1986compilers}.

Para a quarta entrega, focamos no ambiente de execução. O compilador por nós criado terá como linguagem de saída um programa que será executado na máquina virtual conhecida como Máquina de von Neumann (MVN).

Para a entrega atual, buscamos completar a especificação do código gerado pelo compilador e das rotinas do ambiente de execução da nossa linguagem de alto nível, a CZAR.

Como material de consulta, além de sites sobre o assunto e das aulas ministradas, foi utilizado o livro indicado pelo professor no começo das aulas \cite{intro-compiladores}, para pesquisa de conceitos e possíveis implementações.

O documento apresenta a seguir o que foi solicitado na quinta etapa.


\chapter{Questões}
\label{chap:questoes}
	% !TEX encoding = UTF-8 Unicode

A seguir, seguem as respostas às questões propostas pelo professor.

\section{Questão 1}
\label{chap:q1}
	% !TEX encoding = UTF-8 Unicode

\textbf{Quais são as funções do analisador léxico nos compiladores e interpretadores?}

TODO


\section{Questão 2}
\label{chap:q2}
	% !TEX encoding = UTF-8 Unicode

\textbf{Quais as vantagens e desvantagens da implementação do analisador léxico como uma fase separada do processamento da linguagem de programação em relação à sua implementação como sub-rotina que vai extraindo um átomo a cada chamada?}

TODO

	
\section{Questão 3}
\label{chap:q3}
	% !TEX encoding = UTF-8 Unicode

\textbf{Defina formalmente, através de expressões regulares sobre o conjunto de caracteres ASCII, a sintaxe de cada um dos tipos de átomos a serem extraídos do texto-fonte pelo analisador léxico, bem como de cada um dos espaçadores e comentários.}

TODO


\section{Questão 4}
\label{chap:q4}
	% !TEX encoding = UTF-8 Unicode

\textbf{Converta cada uma das expressões regulares, assim obtidas, em autômatos finitos equivalentes que reconheçam as correspondentes linguagens por elas definidas.}

TODO


\section{Questão 5}
\label{chap:q5}
	% !TEX encoding = UTF-8 Unicode

\textbf{Crie um autômato único que aceite todas essas linguagens a partir de um mesmo estado inicial, mas que apresente um estado final diferenciado para cada uma delas.}

TODO


\section{Questão 6}
\label{chap:q6}
	% !TEX encoding = UTF-8 Unicode

\textbf{Transforme o autômato assim obtido em um transdutor, que emita como saída o átomo encontrado ao abandonar cada um dos estados finais para iniciar o reconhecimento de mais um átomo do texto.}

O transdutor obtido a partir da transformação da questão 5 pode ser encontrado no apêndice \ref{app:transdutor}.


\section{Questão 7}
\label{chap:q7}
	% !TEX encoding = UTF-8 Unicode

\textbf{Converta o transdutor assim obtido em uma sub-rotina, escrita na linguagem de programação de sua preferência.}

TODO


\section{Questão 8}
\label{chap:q8}
	% !TEX encoding = UTF-8 Unicode

\textbf{Crie um programa principal que chame repetidamente a sub-rotina assim construída, e a aplique sobre um arquivo do tipo texto contendo o texto-fonte a ser analisado. Após cada chamada, esse programa principal deve imprimir as duas componentes do átomo extraído (o tipo e o valor do átomo encontrado).}

O programa principal que utiliza as sub-rotinas pertencentes ao analisador léxico pode ser encontrada no apêndice \ref{app:codigo-principal}. O código está comentado e seu funcionamento é explicado na questão 9.


\section{Questão 9}
\label{chap:q9}
	% !TEX encoding = UTF-8 Unicode

\textbf{Relate detalhadamente o funcionamento do analisador léxico assim construído, incluindo no relatório: descrição teórica do programa; descrição da sua estrutura; descrição de seu funcionamento; descrição dos testes realizados e das saídas obtidas.}

TODO


\section{Questão 10}
\label{chap:q10}
	% !TEX encoding = UTF-8 Unicode

\textbf{Explique como enriquecer esse analisador léxico com um expansor de macros do tipo \#DEFINE, não paramétrico nem recursivo, mas que permita a qualquer macro chamar outras macros, de forma não cíclica.}

TODO



\chapter{Conclusão}
\label{chap:conclusao}
	% !TEX encoding = UTF-8 Unicode

TODO Conclusão vlassance


\bibliography{bibliografia}

\begin{apendicesenv} % Inicia os apêndices

% Imprime uma página indicando o início dos apêndices
\partapendices

\chapter{Código em C do Analisador Léxico}
\label{app:codigo}
\includepdf[pages={-}]{apendice/codigo_analisador_lexico.pdf}

\end{apendicesenv}

\end{document}
